\subsection{Structure and Meetings}
At the beginning of the project the team had to define a structure for collaboration. We decided to have a flat hierarchy with all members as part of dynamically built development teams and Michael Ruster as the project leader. The role of the project leader was rather informal and was used for bunched external communication with the supervisors and the contest organisers. Decisions were made by the team as a hole via majority decisions. The dynamic groups were built weekly to tackle the newly crafted tasks. In the beginning, we tried some hacking sessions. But we quickly found out that working from home works best for us. Plans and discussions were held together in the weekly meetings on the whiteboard. The rest was left for the teams. In the smaller groups it was much easier to find days (timeslots) where everyone could work.

% TODO: this is from the TOC.
%Dynamic working groups that were built weekly to tackle the newly crafted tasks per week. In the beginning, we also tried some hacking sessions. But we quickly found out that working from home works best for us. Plans and discussions were held together in the weekly meetings (on the whiteboard).
