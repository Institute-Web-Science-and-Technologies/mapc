\subsection{Collaboration Tools}
For collaboration on the code, Github was chosen as a revisionsing system. The included Wiki was used for minutes of our weekly meetings. The provided issue and problem tracking system was used, too. Furthermore VoIP-solutions (Google hangouts and Skype) were used for collaborative programming and reconciliation. Not all problems were transformed into issues. Some were just mentioned and discussed in the Hangouts group chat and then quickly solved after. Eclipse was chosen as the IDE because most of the team members were familiar with it and a plug-in for Jason exists. As it turned out the promising plug-in (http://jason.sourceforge.net/mini-tutorial/eclipse-plugin/) with a mind inspector for debugging couldn't be used because it was very buggy and regularly crashed. As a result the team had to use log files for debugging. 

% TODO: this is from the TOC
%Revisionsing system, Wiki for minutes and issues for problems. VoIP-solutions for collaborative programming. Not all problems were transformed into issues. Some were just mentioned and discussed in the Hangouts group chat and then quickly solved after.
