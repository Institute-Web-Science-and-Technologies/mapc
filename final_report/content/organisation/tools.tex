\subsection[Collaboration Tools]{Collaboration Tools$^{\odot/\circ}$}
For collaboration on the code, GitHub\footnote{\url{https://github.com/} -- last accessed 24.10.2014} was chosen as a revisioning system. % TODO sync dates!
No team member had any prior experience with GitHub.
Some few members had worked with SVN as a revisioning system before.
Nevertheless, we decided to use GitHub as it additionally offers a Wiki and an issue tracker. % TODO does everyone know what a Wiki is? If not, we have a problem for a trustworthy source. http://www.oxforddictionaries.com/definition/english/wiki might be the best I can think of.
A Wiki is an online collaboration tool which enables users to create and edit hypertext pages within their Web browser (cf. \cite{leuf_wiki_2001}).
We used the included Wiki for gathering the minutes of our weekly meetings.
The minutes recorded the attendees, open issues from last meeting, the decisions made in the current meeting as well as a list of assigned tasks to work on until the next meeting.
We did not have a designated minute taker but would rotate alphabetically.
The person to take the minutes was also the one to present our progress at our weekly meetings with our supervisors.
For more complex problems, ideas or bug reports, we used GitHub's issue tracker.
It allowed discussions clearly separated by bug or feature.
This distinct separation was not given for all bug reports as not all problems were transformed into issues.
Instead, many small problems were discussed on our team chat.
For this, we used the instant messenger Google Hangouts\footnote{\url{https://www.google.com/+/learnmore/hangouts/} -- last accessed 24.10.2014} as all team members already had registered a Google account.
The main advantage of Google Hangouts over the issue tracker was that the response time was a lot lower due to its rather informal style.
Some were just mentioned and discussed in the Hangouts group chat and then quickly solved after.
It was also frequently used for short-dated organisational discussions, which would not have fit well into a ticket.
Furthermore VoIP-solutions (Google hangouts and Skype) were used for collaborative programming and reconciliation.

Eclipse was chosen as the IDE because most of the team members were familiar with it and a plug-in for Jason exists.
As it turned out the promising plug-in\footnote{\url{http://jason.sourceforge.net/mini-tutorial/eclipse-plugin/} -- last accessed 24.10.2014} with a mind inspector for debugging couldn't be used because it was very buggy and regularly crashed.
As a result the team had to use log files for debugging.
