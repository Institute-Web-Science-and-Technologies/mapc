\subsection[Yuan Sun]{Yuan Sun$^{\blacktriangle,\diamond/\circ}$}\label{org:sun}
This section lists the work done by Yuan Sun.
Yuan mainly focused on implementing actions of agents in this project.
At the beginning, she started to learn how to use Jason programming language together with Manuel Mittler and Rahul Arora.
Rahul later left the research lab. % I mention this here because this is probably the only time this person is mentioned in this document.

In addition, there was a Java factory class called \texttt{ActionHandler} which was used to instantiate a corresponding \texttt{Action} object according to a given identifier of type string.
It aimed to separate the creation and representation of concrete \texttt{Action} objects.
However, the actions represented as objects were not diverse enough to justify the creation of subclasses per each action.
Hence, a dedicated factory class was not necessary.
Yuan deleted this Java factory class and then encapsulated the progress of object instantiation inside the \texttt{Action} constructor.
The constructor was able to return an \texttt{Action} object according to a given parameter, \texttt{entityName}, so that no complex switch statement would be necessary every time a new \texttt{Action} object was needed.
Therefore, the degree of coupling of the system was decreased.

In the beginning of agent strategies development in Jason, the implementation of different actions was done in a single file.
Yuan rewrote and reorganised some of the code in different files with names corresponding to the roles of different agents.

Yuan implemented initial actions like \texttt{probe}, \texttt{attack}, \texttt{repair} and \texttt{parry} with simple strategies while the other team members were working on map and zone related tasks.
This enabled the other team members sooner to monitor and analyse the functioning of their code.
She also improved the contexts of plans which lead to the execution of these actions.
For example, she added the checks to determine a sufficiently high energy level to all agents before action execution as well as checks for the visibility range limitation for ranged actions.

Moreover, with further project development, Yuan continued implementing some more complex agent strategies after discussing them with other team members.
For instance, previously, sentinel agents only executed the \texttt{parry} action when they were attacked by enemies.
After some discussion, Yuan implemented a better strategy for the these sentinel agents.
If a sentinel agent was occupying a node in order to form a zone, it would execute the \texttt{parry} action when it saw enemies nearby although it had not yet been attacked by the enemies.

In the final technical report, Yuan wrote the sections related to the Belief-Desire-Intention model.
She did a presentation about Belief-Desire-Intention model before the project start.
In the report, she did not only introduce the theory of the Belief-Desire-Intention model (see \autoref{fun:BDI}), but also did some statistics to analyse the use of the Belief-Desire-Intention model in our project (see \autoref{imp:BDI_AS}).
