\subsection[Artur Daudrich]{Artur Daudrich$^{\star}$}
This section lists the work done by Artur Daudrich.
Arturs main influence on our project was the development of ideas and strategies for our map exploration and zoning strategies. At the beginning he implemented the interface between the mapc server and our implementation. This included the passing of percepts from the server to our agents and the passing of actions from our agents to our server. He implemented a basic Java agent class to presave data from the server. This code was at first not used but later in our project again revisited. After implementing the server-client communication and the server interface he started working on percept reception and percept storage in the belief base of our agents. 
He developed some basic Jason plans for exploration which execute actions like \texttt{goto}, \texttt{probe} or \texttt{survey}. Additionaly a first approach of reactive plans were developed by him. These plans reacted to situation where one of our agent has no energy to execute further actions or mets an enemy agent and needs to defend itself or attack it. While working on these plans we got performance issues with our first exploration approach (see \autoref{alg:exploration}). While a group implemented the cartographer agent approach, he adapted the current plans to this new approach. As the cartographer approach did not solve our performance issues Artur came up with the idea of using a Distance-Vector Routing Protocol (see \autoref{alg:map_dv}) and introduced it to the team. While one group tried to implement the Distance-Vector Routing Protocol in Jason, Artur and Manuel Mittler worked on improving storing and communicating agent percepts. They came up with the idea of bypassing the percept passing to the respective agents and sending the information from the server interface directly to the cartographer agent. 
After that Artur worked on his own on a parallel approach for the map component see(\autoref{alg:map_javamap}). His approach was to fully implement the map component in Java to benefit from the speed increase of pure Java. As this approach showed up to boost our performance a lot, the current map team which was working on the Jason map implementation was disbanded and Artur build the new map team together with Michael Sewell. Hence our team always worked in small groups of two or three people, Artur and Michael Sewell worked together on most of the following objectives. They finished the map component in Java and adapted the Jason agent to work with the new map component. Also they implemented a lot of internal actions to allow Jason agents to communicate with the JavaMap.
In the meantime we started to develop ideas for our zoning approach in our weekly team meetings. After defining the outlines of our zoning approach, Artur and Michael Sewell implemented this approach in the JavaMap component. After that Artur was mainly working on the internal actions to allow Jason agents to query zoning and map details. In the final part of our development Arturs task was to fix bugs and implement new features which were associated with the Java part of our code. This includes zoning, exploration, agents classes in Java and the interface to the server.
In this documentation Artur wrote the sections he was mainly involved, which are \autoref{fun:mapc_roles}, \autoref{arc:simulation} and \autoref{alg:exploration}.