\subsection[Artur Daudrich]{Artur Daudrich$^{\star,\circ}$}
This section lists the work done by Artur Daudrich.
Artur's contribution to the project was mainly the development and implementation of ideas and strategies for our map exploration and zoning strategies.
At the beginning he implemented the interface between the MAPC server and our implementation.
This included the passing of percepts from the server to our agents and the passing of actions from our agents to the server.
He implemented a basic Java agent class to presave data from the server, like information about their role, their basic energy value or their basic health value. 
As we at first worked with beliefs in Jason we did not use that until it was later revisited and used for our implementation of the map component in Java.
After implementing the server-client communication and the server interface, Artur started working on percept reception and percept storage in the belief base of our agents.
He developed some basic Jason plans for exploration which execute actions like \texttt{goto}, \texttt{probe} or \texttt{survey}.

Additionally, a first approach of reactive plans was developed by him to allow agents to react on external events.
Such plans would e.g.\ consider a nearby enemy agent, so that our agent would then either defend itself, attack it or avoid enemy agents.
Or they would observe their current energy to determine whether further actions were executable.
As some team members implemented the cartographer agent approach, Artur adapted the plans to this new approach.

We later realised that the cartographer agent approach alone could not solve our performance issues.
Artur then introduced the team to the idea of using a Distance-Vector Routing Protocol as shown in \autoref{alg:map_dv}.
While one group implemented the Distance-Vector Routing Protocol in AgentSpeak(L), Artur and Manuel Mittler worked on improving the storing and communication of agent percepts.
They came up with the idea of bypassing the percept passing to the respective agents.
Instead, the percepts would be sent from the server interface directly to the cartographer agent.

Artur then worked on his own approach for the map component presented in \autoref{alg:map_javamap}.
His idea was to fully implement the map component in Java to benefit from the speed increase of pure Java.
This happened in parallel to the part of the team which further developed map related tasks in AgentSpeak(L).
As Artur's approach boosted our performance a lot, the map team active back then was disbanded.
Artur and Michael Sewell then formed the new map team together.
Our team always worked in small groups of two or three people, Artur and Michael Sewell worked together on most of the following objectives.
They finished the map component in Java and adapted the Jason agents to work with the new map component.
Also, they implemented many internal actions to allow Jason agents to communicate with the JavaMap.

In the meantime the whole team started to develop ideas for our zoning approach in our weekly team meetings.
After defining the outlines of our zoning approach, Artur and Michael Sewell implemented the zone calculation as explained in \autoref{alg:zon_calculation}.
Accompanied by this was the realisation of representing and storing the information about currently built zones in the JavaMap component.

After that Artur was mainly working on the internal actions to allow Jason agents query zoning and map details.
In the final part of our development Artur's task was to fix bugs and implement new features which were associated with the Java part of our code.
This includes zoning, exploration, agents classes in Java and the interface to the server.
In this documentation Artur wrote the sections he was mainly involved, which are \autoref{fun:mapc_roles}, \autoref{arc:simulation} and \autoref{alg:exploration}.
