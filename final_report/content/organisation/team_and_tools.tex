\subsection[Basic Team Organisation and Collaboration Tools]{Basic Team Organisation and Collaboration Tools$^{\odot/\circ}$}\label{org:general}
The topic of this section is the team structure as well as the software we used for working together.
In the first part of this section, the organisation of the team is shown.
It focuses on the distribution of tasks and explains how we worked together.
The second part presents what software tools we have used for working together.
It also remotely discusses the usefulness of the Jason plugin for our tasks.

At the beginning of the project the team had to define a structure for collaboration.
We decided to have a flat hierarchy with all members as part of dynamically built, small development teams.
Michael Ruster was selected as a project leader with his role mainly focussed on organisation.
His tasks are explained in more detail in \autoref{org:0nse}.
Once in every week, the team met in person to discuss the current progress and the upcoming course of actions.
All meetings were recorded by a minute taker.
The minutes logged the attendees, open issues from last meeting, the decisions made in the current meeting as well as a list of assigned tasks to work on until the next meeting.
We did not have a designated minute taker but would rotate alphabetically by surname.
The person to take the minutes was also the one to present our progress at our weekly meetings with our supervisors.
During the weekly team meetings, many of our algorithms were initially developed and discussed.
At the end of each meeting, the worked out tasks for the next meeting were assigned to dynamic groups.
These groups mainly consisted of two or three people with more people working on tasks we found to be more important.
Team members were assigned for a specific task due to personal interest or expertise.
In the beginning, we also tried some hacking sessions, but quickly found out that working from home worked best for us.
This was advantageous because no fixed timeslots needed to be found.
Instead, everybody would work independently when they found the time while staying in contact with the others through chat or voice over IP.
The possibility to share the computer screen contents over IP offered by the voice over IP solutions we used, was of great help.
Therefore, multiple persons could work on the same code at once with one programming and the others reviewing it real-time.
If there was need for reconciliation, for example when tasks of different groups were closely interrelated or dependant on one another, short-termed voice over IP calls were held.
To the end of development, the groups diverged mainly into Artur Daudrich and Michael Sewell working on the Java-side of our code and the rest focussing on implementations in AgentSpeak(L).
The prior group hence concentrated on implementing background calculations like internally modelling and constructing the graph and environment design.
Accordingly, Manuel Mittler, Michael Ruster, Sergey Dedukh and Yuan Sun focussed more on agent programming and developing strategies.

For collaboration on the code, GitHub\footnote{\url{https://github.com/} -- last accessed 24 October 2014} was chosen as a versioning system.
No team member had any prior experience with GitHub.
Some few members had worked with SVN as a versioning system before.
Nevertheless, we decided to use GitHub as it additionally offers a Wiki and an issue tracker.
A Wiki is an online collaboration tool which enables users to create and edit hypertext pages within their Web browser (cf. \cite{leuf_wiki_2001}).
We used the included Wiki for gathering the minutes of our weekly meetings.
GitHub's issue tracker was used for complex problems, ideas or bug reports.
It allowed discussions clearly separated by bug or feature.
This distinct separation was not given for all bug reports as not all problems were transformed into issues.
Instead, many small problems were discussed on our team chat.
For this, we used the instant messenger Google Hangouts\footnote{\url{https://www.google.com/+/learnmore/hangouts/} -- last accessed 24 October 2014} as all team members already had registered a Google account.
The main advantage of Google Hangouts over the issue tracker was that the response time was a lot lower due to its rather informal style.
Some were just mentioned and discussed in the Hangouts group chat and then quickly solved after.
It was also frequently used for short-dated organisational discussions, which would not have fit well into a ticket.
As for voice over IP, we both used Google Hangouts and Skype\footnote{\url{https://www.skype.com/} -- last accessed 24.10.2014} because some team members preferred one application over the other.
Eclipse was chosen as the IDE because all of our team members were familiar with it and a plugin\footnote{\url{http://jason.sourceforge.net/mini-tutorial/eclipse-plugin/} -- last accessed 24 October 2014} for Jason exists.
Besides syntax highlighting for AgentSpeak(L), the plugin also includes a promising mind-inspector for debugging agents and step-based debugging.
Unfortunately, we had to find out that the plugin was not of great use for the Mars scenario.
This was due to the short time frame per simulation step which for one resulted in each agent receiving a lot of information.
Consequently, the mind-inspector often crashed or refreshed the information too fast.
Similarly, step-based debugging was not possible because only the current code execution was halted but not the server simulation.
As a result, debugging was mainly reduced to analysing log files generated from manually added print statements.

In sum, it can be said that we tried to keep our organisation to a basic form.
We made sure that we were able to work well-structured but still quite self-organised and democratic in decision finding to encourage creativity in problem solving.
Analogously, we spent little time on deciding what tools to use.
Instead, we preferred software most of us had already used before or which was the leading free project for the given task.
These approaches allowed us to concentrate on the actual multi-agent system development.
It was necessary due to our inexperience and the scant time we had until the competition.
