\subsection[Michael Ruster]{Michael Ruster$^{\circ}$}
This section lists the work done by Michael Ruster.
He was the designated project leader after he proposed himself and was approved by the rest of the team.
His leadership followed a democratic management style.
Hence, decisions were made by the team as a whole through majority decisions.
This helped team creativity while still keeping a structured working process in contrast to e.g. a laissez-faire approach.
The encouragement of creativity was of special interest due to the inexperience of the team in this field.
It supported new ideas and approaches which could then be immediately discussed and further developed as a group.
An autocratic style on the other hand would have needed an expert in this field for wise delegations as well as him having some means of exerting pressure.

The project leader carried out the external communication with the supervisors and contest organisers.
This includes e.g.\ the writing of the contest participation registration document.
Initiated by Manuel Mittler, Michael worked on the slides for our final presentation which was held by both together.
He ran and monitored the test simulations together with the organisers of the MAPC as well as the final simulations.
Here, he also configured and maintained the server.
Michael was also responsible for configuring the server and running test simulations.
Furthermore, he regularly controlled the quality of the minutes and improved them when necessary.
Similarly, Michael maintained the structure of the GitHub repository by introducing various folders, renaming and moving files.
He wrote the guidelines and the table of contents of this report.
Moreover, he incorporated and adapted the Springer lecture notes in computer science template\footnote{\url{https://www.springer.com/computer/lncs/lncs+authors} -- last accessed 17 November 2014} used for this report.
In the beginning, he also started off with requirements engineering for a more formal approach to the software development.
This effort was discarded by the team as many requirements were only discovered during the development and others changed frequently.
Michael licenced the code after investigating the options given limited by the used external libraries.

In our development phase, Michael was first concerned with map-related tasks.
There, he initially implemented the cartographer agent which is presented in~\autoref{alg:map_cartographer}. % cartographer @ b3122c9215265239f69f649c0ceb474718184147, Java map approach until b3122c9215265239f69f649c0ceb474718184147
He then continued maintaining this approach together with Sergey Dedukh and Manuel Mittler.
When Artur Daudrich proposed the Distance-Vector Routing Protocol approach, Michael Ruster and Michael Sewell implemented, tested and improved the node agents
Both are explained in~\autoref{alg:map_dv}. % 84597446622d336fa2e5926ff21d3173f9518b33 onward
Manuel and Michael Ruster later started working on zone forming, maintenance and destruction. % 6f2874f750fe47ff3cd6546a7fa0e3f3279b2acc
After Manuel had to take a pause due to personal matters, Michael continued implementing the zone logics as presented in~\autoref{alg:zon_finding} and \autoref{alg:zon_roles}.
As agent plans for the zone building phase used various internal actions accessing the JavaMap shown in~\ref{alg:map_javamap}, Michael was also active in fixing bugs in the Java.
We were not able to properly handle the end of a simulation which is indicated by the receipt of the \texttt{SIM-END} message \cite{ahlbrecht_protocol_2014}.
The problem was that it was a complex task to fully reset our local simulation and restoring the initial state of agent knowledge.
Here, Michael implemented a workaround.
It made our multi-agent system shut down completely on receipt of the \texttt{SIM-END} message.
The system was then started anew after giving the MAPC server some time to restart the simulation.
This was done by using a bash shell script.
