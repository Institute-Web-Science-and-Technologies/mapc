The \emph{Multi-Agent Programming Contest} is an annual online programming contest hosted by the Clausthal University of Technology since 2005.
Participation is free to any interested groups, and in former years rewards were given to winning teams in the shape of book vouchers.
This year's MAPC, however, was an \enquote{informal} contest, and no prizes were awarded.
The aim of the MAPC is to promote academic interest in the field of multi-agent systems, that is, systems in which multiple artificial agents have to collaborate to achieve a goal.

The nature of the task in which the agents compete in the MAPC has changed over the years, but since 2011 it has been the same \enquote{Agents on Mars} scenario, which will be described below.
The winner and further rankings of each year's contest are determined by having each group's agent systems face off against the other in a tournament, and awarding points to each team according to their performance in each match.

\subsection{The \enquote{Agents on Mars} scenario$^\dagger$}
The \enquote{Agents on Mars} scenario is the one that has been used in the yearly MAPC since 2011.
In it, two opposing teams of agents are placed on vertices in a randomised graph.
Each vertex in the graph has a value which is used for scoring, and agents can traverse the graph by moving along the edges connecting the vertices.
The \enquote{Agents on Mars} name relates to the fictional background \enquote{story} of the scenario: Man has populated Mars, and must find and occupy wells of water on the surface of the planet and protect them from \enquote{pirates}.

The simulation is turn-based, and each agent can perform one action per turn.
There are 28 agents in each team, and each agent belongs to one of five different agent classes, where the agent's type determines the kind of actions the agent can perform and other values used to further differentiate agent classes.
The goal of each match is to have a higher score than the opponent's team at the end of a predetermined number of steps (400 in the 2014 MAPC).
A high score is achieved by finding localised parts of the graph which contain high-value vertices, surrounding these \enquote{zones} with one's own agents, and protecting them from enemy agents' attacks.
The full background story, as well as a more detailed official description of the scenario, can be found in the scenario description provided by the MAPC organisers~\cite{ahlbrecht_mapc_2014}.

\subsection{The \emph{MAKo} (Multi-Agents Koblenz) team$^\dagger$}
The German University of Koblenz-Landau participated in the 2014 MAPC with a small team of graduate students in the scope of a research lab.
The students who participated in research lab until its conclusion were Artur Daudrich, Sergey Dedukh, Manuel Mittler, Michael Ruster, Michael Sewell and Yuan Sun.
The research lab spanned a single semester and consisted of an initial seminar phase and the longer project phase, where the students designed and implemented the multi-agent architecture used to participate in the MAPC.
