Multi-agent systems are an ongoing field of study with diverse applications like modelling stock markets~\cite{lebaron_building_2002} to human immune systems~\cite{folcik_basic_2007}.
Agents are embedded in scenarios which describe the environments and assign the agents specific tasks.
Some tasks can be fulfilled by agents working independently while others need them to work cooperatively.
There are also scenarios in which agents work competitively.

The \emph{Multi-Agent Programming Contest} is an annual online programming contest trying to promote academic interest in the field of multi-agent systems.
Its scenarios encourage cooperative task solving while competing against the agents of another participants' team.
Participation is free to any interested groups and in former years rewards were given to winning teams in the form of book vouchers.
The contest of 2014, however, was an \enquote{informal} contest and no prizes were awarded.

This report documents the participation in this year's contest by a small team of graduate students in the scope of a research lab of the University Koblenz~$\cdot$~Landau.
Our team participated under the name \emph{MAKo} (short for: Multi-agents Koblenz) and scored the second place.
The document is structured as follows.
First, \autoref{fun} presents the fundamentals for this report and the later described development of our own multi-agent system.
It presents the Multi-Agent Programming Contest in more detail, introduces multi-agent programming concepts and discusses various agent programming languages.
\autoref{alg} describes important algorithms and strategies we have developed in order to compete in this year's contest.
The section mainly investigates our implementation from a multi-agent point of view in context of the scenario.
\autoref{imp} however illustrates a micro-level view on a general agent within its reasoning cycle determined by our chosen agent programming language Jason and our implementation.
Next, \autoref{org} describes our team organisation as well as each member's individual tasks.
Finally, happenings of the two competition days are described in \autoref{con} together with conclusions we drew from them as well as our project as a whole.
