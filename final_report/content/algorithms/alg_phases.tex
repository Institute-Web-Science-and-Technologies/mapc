\subsection[Simulation Phases]{Simulation Phases$^{\star,\circ}$}\label{arc:simulation}
This section illustrates our match strategy roughly from a high-level point of view.
Our general approach for the simulation was to split it up into two phases, namely an exploration phase and a zoning phase.
In the exploration phase, we tried to explore the map as quickly and complete as possible.
Throughout the zoning phase, agents would look for zones, form and defend them.
The exploration phase is explained in this section in more detail, while the zoning phase is covered in separate section due to its complexity.

Basically all agents were used for exploration, but we used different priorities for every role.
The highest priority was always to use the role defining ability if applicable.
For example if the Saboteur agent could attack any enemy agent, it attacked.
If a Repairer agent could repair some damaged agent from our team, it repaired.
When the highest priority action was not applicable each agent decided which action to do autonomously based on the information it got from the map component or its percepts.
The component itself centrally stored all information about the map and is presented in \autoref{alg:map_javamap}.
We distinguished three group types of exploring agents.

The first group consisted only of Explorer agents.
Their highest priority was to get vertex values by probing.
Secondarily they used the \texttt{survey} action to explore the map only if they came across vertices which were not surveyed.
They were not actively searching for or going to not surveyed vertices.
Instead, they moved somewhat circular towards vertices which were not surveyed yet as further described in \autoref{alg:agentstrategies}.

The second group was comprised of Saboteur agents.
We followed a very aggressive strategy and aimed to attack and disturb the enemy as much as possible.
Consequently, we did not want to distract our Saboteur agents by exploring the map.
Saboteur agents would only explore the map if they did not know of an active enemy somewhere on the already explored map.
As part of our aggressive strategy, there was one Saboteur agent which would even then not start exploring.
Instead, it would try to increase its visibility radius to find more distant enemies.
This is explained in more detail in \autoref{alg:agentstrategies}.
The Saboteur agents stayed with this strategy throughout the whole simulation and not only the exploration phase.

The third and last group consisted of the remaining agents, which were the Repairer agents, the Sentinel agents and the Inspector agents.
These agents were used mainly for exploring the map.
They were coordinated by querying the map component for the next unexplored map area. We distinguished between explored areas, which are map areas where we know the weights of all vertex edges, and unexplored areas where we do not know the weights of all vertex edges.
If agents of the third group came to a vertex that had unknown edge weights to at least one neighbour vertex, they used the survey action to get the information as percepts.
These weights were then passed to the map component and they repeated the exploring, by querying for the next not surveyed vertex and going there.
To prevent multiple agents from going to the same vertex and exploring the same area, the map component used an internal locking mechanism.
For zoning we were not interested in a full coverage of the map.
This was because a full map exploration would have consumed a lot more steps while bringing only little improvement to the knowledge about it as a whole.
Of course it could be that at some point during the simulation we gain a full coverage of the map, but it was not a criteria to switch to the second phase in the simulation.

At some point in the simulation, around step 100, there were more agents available for exploring than vertices that needed to be surveyed. Responsible for this could be an almost fully explored map or a bottleneck vertex which is the only connection to other parts of the map.
Agents without an assigned vertex would be idle and waiting for the next step.
As we assumed that our agents would always be evenly placed on the map at the beginning of the simulation, we did not pursue a solution for the bottleneck vertex.
Instead, we decided to remove every agent from the exploring agent team which could not be assigned an unsurveyed vertex and put it in the zoning team.
In the zoning phase, all agents which were finished with their exploration phase were used to build zones.
This included the Explorer agents but excluded the Saboteur agents, because they were following their own aggressive strategy.
A detailed description of the zoning phase is given in \autoref{alg:zoning}.
