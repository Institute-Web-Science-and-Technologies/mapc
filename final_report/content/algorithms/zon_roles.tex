\subsubsection{Zoning Roles}\label{alg:zon_roles}
This subsection describes the two roles exclusive to zoning and their associated tasks and duties throughout the lifecycle of a zone. These roles are those of a \emph{coach} and a \emph{minion}. Each zone is built by one coach and a varying amount of minions. Minions are agents which are dedicated to build a zone by obeying their coach's orders. Every agent may only be part of one zone at a time. % The last sentence is probably redundant.
The roles are assigned when a concrete zone is about to be built. Zoning agents keep either of these roles until the zone is broken up or they have to leave it. The roles regulate the agents' behaviour throughout the time they spend in a zone.

% This subsection describes the communication hierarchy during active zoning until its breakup

Before looking at border cases, an ideal case of a zone lifecycle is presented. There, the zone negotiation described in subsection~\ref{alg:zon_formation} ends with all agents knowing about the same best zone. This zone was found by one agent which will then become the zone's coach. Next, the coach informs the agents which will be part of the zone where to go to. On receipt of this message, the agents become minions and move to their designated node. The coach will also have to move to his node, which happens to be the centre node of the zone. Furthermore, the coach will unregister himself and all his minions to indicate their unavailability to build any other zone.
In a zone, minions serve no other purpose than to occupy their designated node. If a minion becomes disabled, he has to move towards a repairer. Due to this, he has to leave his node. Therefore, the zone can no longer exist in its original form. In such a case, a minion has to inform his coach about his departure. The coach must then tell all his other minions that the zone can no longer be maintained. Consequential, all affected agents drop their role and restart looking for zones as illustrated in subsection~\ref{alg:zon_formation}.

In reality, the zoning process is asynchronous. Therefore, it is likely that some agents start looking for a zone when others have nearly finished. Since a result, there can be multiple groups of agents with different knowledge about which zone would currently bring the highest score per agent. Each group could then be expecting a different agent to become a coach. This interferes with the assumptions that each agent may only be in one zone and have only one role at a time. As a solution, coaches do not only inform their minions about where to move to. Instead, they also transmit the per agent score of the zone they want to build together with this agent. Any agent can then compare the received zone score with the zone it wanted to build before. If it is higher, he must inform the coach of his former zone or his minions if he had been the coach himself. In case that the proposed zone's score is lower than the zone the agent intended to form, he must inform the coach who just proposed the new zone. Said coach will then have to inform all his minions that his zone is not going to be built.

Besides coaches and minions, there are also other agents who might be looking for a zone but will not be part of the one which will be built. Such an agent should not simply wait until a new zone is built. Instead, he should look for any highly valuable node in his surrounding which is not yet occupied by anyone. The range to look for such a node is the same as with the range for finding a zone in the agents neighbourhood presented in section~\ref{alg:zon_construction}. It is increased after every zone finding process which does not result in a zone where the agent is part of. The idea is that with a wider range, the probability to find a highly valuable zone increases. Additionally, the agent will likelier move farther away from his position in case he is not part of the zone to be built. This should further ensure that the same zones are only proposed multiple times as best zones if they have a very high per agent score.

We assume due to our colouring algorithm for zone finding that a node within a zone will be occupied by at most one agent. %TODO it won't be called a colouring algorithm later on.
Then, any enemy agent close by a zone endangers it. This is because a zone may not spread across an enemy inside of it~\cite{ahlbrecht_mapc_2014}. % p.12
Furthermore, enemy saboteurs can disable zoning agents, which similarly destroys the zone in its original form~\cite{ahlbrecht_mapc_2014}. % p.11
Hence, coaches check once per step whether an enemy agent is close to the zone. If this is the case, the coach broadcast a message to all saboteurs to come and defend the zone. The saboteurs bid for this with the closest saboteur to the zone's centre winning. He will then move towards the enemy to disable him. If the coach detects in a next step that the enemy moves away from the zone, he will cancel the zone defence through another broadcast.
