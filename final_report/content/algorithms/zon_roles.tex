\subsubsection[Zone Building Roles and the Lifecycle of a Zone.]{Zone Building Roles and the Lifecycle of a Zone.$^\circ$}\label{alg:zon_roles}
This subsection describes the two roles exclusive to zone building. It covers the roles' associated tasks and duties throughout the lifecycle of a zone as well as the lifecycle itself. These roles are those of a \emph{coach} and a \emph{minion}. Each zone is built by one coach and a varying amount of minions. Minions are agents which are dedicated to build a zone by obeying their coach's orders. Every agent may only be part of one zone at a time. % The last sentence is probably redundant.
The roles are assigned when the zone finding process has ended and a concrete zone is about to be built. Agents keep either of these roles until the zone is broken up or they have to leave it. The roles regulate the agents' behaviour throughout the time they spend in a zone.
% This subsection describes the communication hierarchy during active zoning until its breakup

Before looking at border cases, an ideal case of a zone lifecycle is presented. There, the zone finding process described in \autoref{alg:zon_finding} ends with all agents knowing about the same best zone. This zone was found by one agent which will then become the zone's coach. Next, the coach informs the agents which will be part of the zone where to go to. On receipt of this message, the agents become minions and move to their designated node. The coach will also have to move to its node, which happens to be the centre node of the zone. Furthermore, the coach will unregister itself and all its minions to indicate their unavailability to build any other zone.
In a zone, minions serve no other purpose than to occupy their designated node. If a minion becomes disabled, it has to move towards a repairer agents. Due to this, it has to leave its node. Therefore, the zone can no longer exist in its original form. In such a case, a minion has to inform its coach about its departure. The coach must then tell all its other minions that the zone can no longer be maintained. Consequently, all affected agents drop their role and restart looking for zones as illustrated in \autoref{alg:zon_finding}.

In reality, the zone finding process is asynchronous. Therefore, it is likely that some agents start looking for a zone when others have nearly finished. As a result, there can be multiple groups of agents with different knowledge about which zone would currently bring the highest score per agent. Each group could then be expecting a different agent to become a coach. This interferes with the assumptions that each agent may only be in one zone and have only one role at a time. To solve this problem, coaches do not only inform their minions about where to move to. Instead, they also transmit the per agent score of the zone they want to build together with this agent. Any agent can then compare the received zone score with the zone it wanted to build before. If it is higher, it must inform the coach of its former zone or its minions if it had been the coach itself. In case that the proposed zone's score is lower than the zone the agent intended to form, it must inform the coach who just proposed the new zone. Said coach will then have to inform all its minions that its zone is not going to be built.

Besides coaches and minions, there are also other agents who might be looking for a zone but will not be part of the one which will be built. Such an agent will have to start a new zone finding process. Prior to that though, it will look for any highly valuable node in its surrounding which is not yet occupied by anyone and move there. The range to look for such a node is the same as the range for finding a zone in the agent's neighbourhood presented in \autoref{alg:zon_calculation}. It is increased after every zone finding process which does not result in a zone where the agent is part of. The idea is that with a wider range, the probability to find a highly valuable zone increases. Additionally, the agent will likelier move farther away from its position in case it is not part of the zone to be built. This should further ensure that zones are only proposed multiple times as best zones if they have a very high per agent score.

We assume due to our zone calculation algorithm that a node within a zone will be occupied by at most one agent. %TODO it won't be called a colouring algorithm later on.
Then, any enemy agent close to a zone endangers it. This is because a zone may not spread across an enemy inside of it~\cite{ahlbrecht_mapc_2014}. % p.12
Moreover, enemy saboteur agents can disable agents inside a zone, which similarly destroys the zone in its original form~\cite{ahlbrecht_mapc_2014}. % p.11
Hence, coaches check once per step whether an enemy agent is close to the zone. If this is the case, the coach broadcasts a message to all saboteur agents to come and defend the zone. Saboteur agents which are not already defending a zone bid for this. The saboteur agent closest to the zone's centre will win the bidding. It then moves towards the enemy to disable it. If the coach detects in a next step that the enemy moved away from the zone, it will cancel the zone defence. The coach does so by using another broadcast as it does not know which saboteur agent was selected to defend the zone.

Explorer agents will still be probing when the first agents start looking for zones. Therefore, the most valuable zones may change with more and more nodes being probed. To prevent that agents build a zone once and stay there for ever if no agents attack them, zones will be split up periodically. The periodic trigger is linked to the overall steps of the simulation and not the lifetime of each zone respectively. Consequently, agents from different zones will have to restart looking for a zone at the same time. In addition to allowing new zones to be build which take the information of the newly probed nodes into account, this also allows for agents to start the zone finding process in a less asynchronous fashion.
% TODO: proper ending of this section
