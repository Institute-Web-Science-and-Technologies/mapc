\subsection[Agent Specific Strategies]{Agent Specific Strategies$^{\dagger,\circ}$}\label{alg:agentstrategies}
Each of the five different agent types (or \emph{roles}) in the MAPC scenario --- Explorers, Repairers, Saboteurs, Sentinels and Inspectors --- has different capabilities in terms of the actions they can perform.
Thus, they must each act according to role-specific strategies and tactics in order for the team to perform well.
This section will give a short overview of how different agent types behave differently from each other.
\begin{description}
    \item[Explorer] agents are the only ones who can execute the \texttt{probe} action.\todo{\textbf{@msewell}: \enquote{For all assumptions or design choices you make, add a justification and hint to possible problems} - Matthias}
        Vertices must be probed in order to learn their value, which is critical for building zones with high scores.
        Accordingly, an Explorer agent will spend most of its time seeking out, moving towards and finally probing vertices whose values are not yet known.
        Having multiple explorer agents move towards the same not yet probed vertex is in most cases suboptimal.
        Since there are 6 Explorer agents in a team, special care is taken to prevent this as described in \autoref{alg:exploration}\todo{update this reference to wherever we explain why multiple agents won't run onto the same node}.

        In our implementation, we are generally interested in the value of all vertices and hence would want to have all vertices probed.
        Yet, we prioritise vertices in a specific way which we call \emph{cluster probing}.
        Therefore, an Explorer agent's probing will rather be circular than e.g.\ a straight line.
        The motivation behind this is that our algorithm for zone calculation as presented in \autoref{alg:zon_calculation} puts agents around a centre vertex in a circular manner and with a maximum distance of two edges away.
        By having cluster probing mimic the circular shape of the zones calculated by our algorithm, we enable our agents to quicker find and establish high-value zones.
        To achieve this, our probing algorithm prioritises not yet probed vertices first by distance, then by the number of edges they share with already probed vertices.
        The result is that the Explorer agents' movement is similar to a spiral pattern, provided the Explorer agents are not disturbed by e.g.\ nearby enemy agents.
        Explorer agents do not stop this probing pattern until they can no longer find any not yet probed vertices.
    \item[Repairer] agents are the only agents who can perform the \texttt{repair} action for restoring health of disabled agents.
        Disabled agents can only move and recharge and cannot be used in building zones.
        Therefore, a team loses out on possible points for every disabled agent in the team.
        So to achieve a high score it is essential to quickly repair damaged agents.
        In our implementation, Repairer agents' actions are prioritised so that they will attempt to repair any disabled friendly agent in their visibility range.
        It is the \enquote{job} of the disabled agents to find and move towards the closest friendly Repairer agent.
        If a Repairer agent is aware of a friendly disabled agent outside of its visibility range, and the Repairer agent is currently not used for zoning, however, then the Repairer agent will also move towards the disabled agent\todo{\textbf{@sdedukh}, \textbf{@msewell}: \autoref{alg:repairing} says, this is only the case throughout exploration phase. Code-wise, I did not find any clue that repairers would move towards disabled agents at any time. What's correct here? Update both sections accordingly.}.
        More detail on the process of repairing is given in \autoref{alg:repairing}.
    \item[Saboteur] agents are the only agents that can disable enemy agents using the \texttt{attack} action.
        In our implementation, a Saboteur agent's role is very aggressively defined.
        The prioritisation is therefore to attack all non-disabled enemies within the Saboteur agent's visibility range.
        If the Saboteur agent does not see such an enemy, it will try to find one and then move towards it to attack it.

        Throughout our development, Saboteur agents were the only agent types which would use the \texttt{buy} action.
        Our initial strategy would allow each Saboteur agent to buy one upgrade to extend its visibility range.
        Doing so increases the probability for successful ranged attacks \cite{ahlbrecht_mapc_2014} and would support the Saboteur agents' offensive strategy.
        We decided to try out other buying strategies by having our agents play matches against copies of themselves with the copies using different strategies for buying upgrades.
        Although it was a brief, empirical and rather informal testing approach, we discovered a strategy which would lead to persistently higher scores.
        Instead of each Saboteur agent buying a visibility range upgrade once, we would allow a special Saboteur agent to buy multiple upgrades.
        We called this agent the \emph{artillery agent} due to its ability to successfully attack agents multiple hops away.
        The artillery agent decided to buy upgrades whenever it would not have a non-disabled enemy within its visibility range and buying was possible given the amount of available money.
        Thus, we were able to outperform copies of our agents which were using our more conventional approach of upgrade buying.

        Our artillery agent can buy upgrades for its maximum energy, visibility range, maximum health or strength.
        The kind of upgrade which it buys, depends on the relative improvement this upgrade will bring to the upgrade-specific \enquote{module}.
        For example, if the artillery agent has a maximum health of $3$ and a visibility range of $1$, then increasing the maximum health to $4$ would be an improvement of \SI{33}{\percent}.
        However, increasing the visibility range from $1$ to $2$ would be an improvement of \SI{100}{\percent}.
        Thus in this example, the artillery agent will choose to buy an upgrade for its visibility range.

        We also call in Saboteur agents to defend a zone if it gets attacked by an enemy agent as mentioned in \autoref{alg:zon_roles}.
    \item[Sentinel] agents do not have a unique action that they can perform.
        Their strength is that they start with a visibility range of 3 by default, which is the highest of all the agent types.
        This is useful during exploration and to be warned of incoming enemy agents.
        Yet, we do not use any Sentinel specific logic in our implementation worth mentioning.
    \item[Inspector] agents are uniquely able to perform the \texttt{inspect} action.
        We consider it important to inspect every enemy agent at least once during every match to learn and store that agent's role.
        This is necessary to know which enemy agents are Saboteur agents, so that our agents can avoid them.

        Furthermore, inspecting enemy agents is rewarded with achievement points.
        They are not rewarded for performing \texttt{inspect} actions on an enemy agent more than once.
        The only use case for inspecting an enemy agent more than once during a single match would be to learn if they have bought any upgrades since the last time they were inspected.
        But this year's and last years' matches showed that buying an upgrade for an agent is a rare occurrence during the actual contest.
        Therefore, re-inspect is not of much interest and inspecting each enemy agent only once could be sufficient.
        In our implementation, however, we toggle an enemy agent to be ready to be inspected again 50 turns after it was last inspected.
\end{description}
