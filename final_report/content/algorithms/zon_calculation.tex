\subsubsection{Zone Calculation}
\label{alg:zon_calculation}
The graph colouring algorithm used by the MAPC server to determine occupied zones is described in detail in the MAPC 2014 scenario description~\cite{ahlbrecht_mapc_2014} and will not be explained again here.
Due to the way the server-side colouring algorithm works, placing $n$ agents on the map so that they establish the highest possible zone value per step is anything but straight-forward.
Even for $n = 1$, a single agent placed on an articulation point in the graph can establish a high-value zone if there are no enemy agents in either subgraph that it splits the map into.
\autoref{fig:articulation_points} shows an example.
\begin{figure}
  \centering
  \includegraphics[width=\linewidth]{images/articulation_points}
  \caption{By occupying an articulation point, a single agent can ostensibly establish a a high-scoring zone---provided that there are no enemy agents inside the subgraph that is split off from the main graph.}
  \label{fig:articulation_points}
\end{figure}
To position themselves in an optimally-scoring way, agents could run the same algorithm locally to calculate the agent placement that will lead to the highest total sum of zone scores in each step by trying every possible permutation.
However, the number of ways to place $n$ agents on $k$ nodes is $C \left (n+r-1,r-1\right )= \frac{\left(n+r-1 \right )!}{n!\left(r-1 \right )!}$, a number that increases rapidly with $n$ and $k$.
In particular, there are $C \left (28+600-1,600-1 \right ) =\num{3.7463887025070038e+48}$ ways to place 28 agents on 600 nodes, which were the numbers used in the 2014 competition---far too many to calculate in real-time.
Finding an algorithm that calculates high-scoring zones in a limited computation time is one of the major challenges of the MAPC competition.
Our team developed a heuristic algorithm to calculating zones that will be explained below.
The goal is to find for every node in the graph a placement of agents around that node such that:
\begin{itemize}
  \item All of the centre node's one-hop neighbours (those nodes directly connected to the center node through a single edge) will be included in the zone.
  \item Agents can only be placed on the centre node's two-hop neighbours, which are those nodes that are be connected to the centre node through a minimum and maximum of two edges.
  \item The constructed zone's value per agent should be high.
        Ideally, it would be maximal, but the heuristic we use doesn't guarantee this.
\end{itemize}
\autoref{fig:zones} shows some examples of zones that are found using our heuristic algorithm.
\begin{figure}
  \centering
  \subcaptionbox{This zone was calculated for a centre vertex that only has a degree of 1, i.e.\ that is a leaf vertex.
                 Generally, it is preferable to place an agent on the cut vertex that leads to a leaf vertex rather than the leaf vertex itself, as this would establish at least a zone of equal size, and possibly larger.
                 \label{fig:zones_1}}[.49\linewidth]{\includegraphics[width=.49\linewidth]{images/zone1.png}}
  \subcaptionbox{Here, the centre vertex has a degree of 3, and the calculated zone remains compact with only two additional agents used.
                 A lot of optional agent positions remain.
                 \label{fig:zones_2}}[.49\linewidth]{\includegraphics[width=.49\linewidth]{images/zone2.png}}
  \\
  \subcaptionbox{A zone where the centre vertex has a degree of 5, and the zone uses a total of 4 agents.
                 \label{fig:zones_3}}[.49\linewidth]{\includegraphics[width=.49\linewidth]{images/zone3.png}}
  \subcaptionbox{A zone where the centre vertex has a degree of 7, and the zone uses a total of 9 agents.
                 \label{fig:zones_4}}[.49\linewidth]{\includegraphics[width=.49\linewidth]{images/zone4.png}}
  \caption{Four examples of zones calculated by the heuristic algorithm described in \autoref{alg:zon_calculation}.
           The green squares and triangles represent the placement of agents, where the triangle is the agent on the center node.
           Nodes marked with a small green circle are optional agent positions that can be used to expand the zone if there are agents left over at the end of the zone building, as described in \autoref{alg:zon_finding}.
           The green-colored area represents the zone that is established by the given agent placement.}
  \label{fig:zones}
\end{figure}
Every node in the graph is represented by a Java Node object, and the calculated zone is stored as a field of that object.
The zone calculation is (re)-triggered every time a node in the node's two-hop neighbourhood (so within the ambit of the zone we're trying to calculate) is discovered during map exploration or changes its known value when it is probed by an Explorer agent, as these are the events that can lead to the position of agents that construct the zone changing.
The steps of the algorithm are best detailed graphically, as in \autoref{fig:coloring}.
\begin{figure}
  \centering
  \subcaptionbox{The initial step.
                 All nodes within the centre node's two-hop neighbourhood are considered optional agent positions at this point.
                 Note that this also includes those nodes that are at distance 1 from the centre node, even though we ideally only want to place agents on nodes that are 2 edges away.
                 There are some possible fringe cases where it is impossible to fulfill the criteria of having the entire one-hop neighbourhood in the zone by only placing agents on the two-hop nodes, however, and so all the nodes in the one-hop neighbourhood must initially be considered for agent placement.
                 \label{fig:coloring1}}[.49\linewidth]{\includegraphics[width=.49\linewidth]{images/coloring1.png}}
  \subcaptionbox{The algorithm proceeds to add all two-hop nodes that are connected to two or more one-hop nodes to the list of definitive agent positions, denoted by the green colour.
                 Afterwards, those two-hop nodes connected either directly to a \enquote{green} node or indirectly connected to a \enquote{green} node through a single one-hop node are removed from the list of possible agent positions (yellow).
                 The reasoning behind this is that those one-hop nodes adjacent to the now green nodes are already definitely included in the zone, and the removed nodes do not contribute towards that goal.
                 \label{fig:coloring2}}[.49\linewidth]{\includegraphics[width=.49\linewidth]{images/coloring2.png}}
  \\
  \subcaptionbox{In the next step, \enquote{bridges} are discovered in the list of remaining yellow two-hop neighbours.
                 A bridge is considered to be a connected triple of nodes where one of the nodes is directly connected to the other two.
                 If such a bridge exists in the list of remaining two-hop nodes, all three nodes can be included in the zone by placing an agent on either end of the chain and leaving out the in-between node.
                 Since three nodes can be captured in the zone for the \enquote{cost} of two agents, we consider this a good exchange to make.
                 \label{fig:coloring3}}[.49\linewidth]{\includegraphics[width=.49\linewidth]{images/coloring3.png}}
  \subcaptionbox{In the last step, the algorithm checks if all one-hop neighbours are connected to an agent position (green nodes).
                 This is most frequently the case, as in this example, but not always.
                 If a remaining, unconnected one-hop is found, we check if it is connected to one or more \enquote{yellow} two-hop nodes.
                 If that is the case, we choose the neighbouring two-hop with the highest node value and add it to the list of agent positions.
                 If no such two-hop node is found, we add the unconnected one-hop node to the list of agent positions---this is the only case where a one-hop node can be added to the list of agent positions.
                 Finally, we include the center node in the list of agent positions and colour all remaining yellow one-hops red.
                 Any yellow nodes that remain are saved as possible agent positions that could be used to extend the zone by otherwise idle agents, but unlike the green nodes are not required to establish the initial smallest zone that we calculated.
                 \label{fig:coloring4}}[.49\linewidth]{\includegraphics[width=.49\linewidth]{images/coloring4.png}}
  \caption{The zone calculation algorithm shown in four steps.
           Nodes are coloured differently according to their current state as the algorithm progresses.
           Green nodes are nodes that an agent must be placed on, red nodes are those where placing an agent would be redundant because it does not extend the zone.
           Yellow nodes denote notes where agents could be placed to extend the zone, but are not considered optimal in the eyes of the algorithm.
           Numbers shown next to nodes represent their edge distance from the centre node.}
  \label{fig:zones}
\end{figure}
While we consider our algorithm to find zones of acceptably high zone values per agent, it can easily be shown to be suboptimal.
For one, it only considers nodes within the two-hop neighbourhood of the centre node, and it is not difficult to think of possible graph structures where a different agent placement would lead to a better zone.
For example, if one of the remaining yellow nodes in \autoref{fig:coloring4} were an articulation point whose inclusion in the list of agent positions would add more than that single node to the zone, this would not be discovered by the algorithm.
