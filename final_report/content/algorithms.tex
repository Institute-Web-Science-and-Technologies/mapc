\subsection{General Strategy Overview}
This could also be an introductionary text which motivates the following subsections.

%%%%%%%%%%%%%%%%%%
\subsection{DSDV}
What is it? How is it used in our context? What are advantages we gain from it? What is problematic (speed loss)?

%%%%%%%%%%%%%%%%%%
\subsection{Exploration}
How do agents move around during the exploration phase?

%%%%%%%%%%%%%%%%%%
\subsection{Zone Forming}
% TODO: update this according to how the subsection will be in the end
Zoning is the most important part in the MAPC Mars scenario~\cite{ahlbrecht_mapc_2014}.% p.3
It describes the process of agents occupying nodes in a way that they enclose a subgraph. In subsection~\ref{alg:zon_roles} we introduce two additional agent roles which are assigned during zoning. Said roles define the tasks and duties of an agent in this phase.

For our approach, zoning should take place after the map exploration phase. This should ensure that enough information about the map has been gathered to calculate high valuable zones close to the agents' current positions. The algorithm for finding these zones and determining which agents have to occupy which nodes is presented in subsection~\ref{alg:zon_colouring}.

The process of forming a zone and the associated agent communication is presented in the last subsection~\ref{alg:zon_formation}. It features the assignment of zone roles to agents. Furthermore, it is illustrated what zone is to be built and what the agents have to do to achieve this.

\subsubsection{Colouring Algorithm for Zone Finding}\label{alg:zon_colouring}
In some way, we try to find local maxima to build small zones with as few agents as possible. How do we find zones? How do we find out how many agents we need? Explain that extending zones describes how the score resulting from an active zone can be increased by using idle agents. How do we determine what additional spots for zone extensions exist?
Present our colouring algorithm and the concept of a centre node.


\subsubsection{Zone Negotiation}\label{alg:zon_formation}
Zoning happens asynchronously. Who registers when for zoning? How do they unregister?
How do agents decide what zone to form? Which agents become coaches and which become minions? How do agents know which node they have to occupy?

\subsubsection{Zoning Roles}\label{alg:zon_roles}
This subsection describes the two roles exclusive to zoning and their associated tasks and duties throughout the lifecycle of a zone. These roles are those of a coach and a minion. They are assigned when a concrete zone is about to be built. Zoning agents keep either of these roles until the zone is broken up or they have to leave it. The roles regulate the agents' behaviour throughout the time they spent in a zone. Each zone is built by one coach and a varying amount of minions. Minions are agents which are dedicated to build a zone by obeying their coach's orders. Every agent may only be part of one zone at a time.
% This subsection describes the communication hierarchy during active zoning until its breakup

Before looking at border cases, an ideal case of a zone lifecycle is presented. There, the zone negotiation described in subsection~\ref{alg:zon_formation} ends with all agents knowing about the same best zone. This zone was found by one agent which will then become the zone's coach. Next, the coach informs the agents which will be part of the zone where to go to. On receipt of this message, the agents become minions and move to their designated node. The coach will also have to move to his node, which happens to be the centre node of the zone.

We assume due to our colour algorithm for zone finding that a node within a zone will be occupied by at most one agent. Then, any enemy agent endangers a zone. This is because a zone may not spread across an enemy inside of it~\cite{ahlbrecht_mapc_2014}. % p.12
Furthermore, enemy saboteurs can disable zoning agents, which similarly destroys the zone in its original form~\cite{ahlbrecht_mapc_2014}. % p.11
Hence, coaches check once per step whether an enemy agent is close to the zone. If this is the case, the coach broadcast a message to all saboteurs to come and defend the zone. The saboteurs bid for this with the closest saboteur to the zone's centre winning. He will then move towards the enemy to disable him. If the coach detects in a next step that the enemy moves away from the zone, he will cancel the zone defence through another broadcast.
% below here are notes for things to mention:
% There are coaches and minions. Coaches command minions. There are also agents who don't get assigned a specific role and try to find and go to a well. There is a strict hierarchy between the roles.
% Furthermore, zone breakups are a process that is different depending on the role of the executing agent. Hence breakups can be introduced and explained here.

%%%%%%%%%%%%%%%%%%
\subsection{Agent Specific Strategies}
How are the agents specialised? Explorers keep probing for long. Inspectors probe enemies so that we can avoid saboteurs. Saboteurs are attacking and can be called for zone defence. Disabled agents communicate with repairers and approach them.
