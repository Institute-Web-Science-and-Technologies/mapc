\todo{Write this part}
This could also be an introductory text which motivates the following subsections.

%%%%%%%%%%%%%%%%%%
\subsection[Simulation Phases]{Simulation Phases$^{\star,\circ}$}\label{arc:simulation}
This section illustrates our match strategy roughly from a high-level point of view.
Our general approach for the simulation was to split it up into two phases, namely an \emph{exploration phase} and a \emph{zoning phase}.
In the exploration phase, we tried to explore the map as quickly and complete as possible.
Throughout the zoning phase, agents would look for zones, form and defend them.
The exploration phase is explained in this section in more detail, while the zoning phase is covered in separate section due to its complexity.

Basically all agents were used for exploration, but we used different priorities for every role.
The highest priority was always to use the role defining ability if applicable.
For example if the Saboteur agent could attack any enemy agent, it attacked.
If a Repairer agent could repair some damaged agent from our team, it repaired.
When the highest priority action was not applicable each agent decided which action to do autonomously based on the information it got from the map component or its percepts.
The component itself centrally stored all information about the map and is presented in \autoref{alg:map_javamap}.
We distinguished three group types of exploring agents.

The first group consisted only of Explorer agents.
Their highest priority was to get vertex values by probing.
Secondarily they used the \texttt{survey} action to explore the map only if they came across vertices which were not surveyed.
They were not actively searching for or going to not surveyed vertices.
Instead, they moved somewhat circular towards vertices which were not surveyed yet as further described in \autoref{alg:agentstrategies}.

The second group was comprised of Saboteur agents.
We followed a very aggressive strategy and aimed to attack and disturb the enemy as much as possible.
Consequently, we did not want to distract our Saboteur agents by exploring the map.
Saboteur agents would only explore the map if they did not know of an active enemy somewhere on the already explored map.
As part of our aggressive strategy, there was one Saboteur agent which would even then not start exploring.
Instead, it would try to increase its visibility radius to find more distant enemies.
This is explained in more detail in \autoref{alg:agentstrategies}.
The Saboteur agents stayed with this strategy throughout the whole simulation and not only the exploration phase.

The third and last group consisted of the remaining agents, which were the Repairer agents, the Sentinel agents and the Inspector agents.
These agents were used mainly for exploring the map.
They were coordinated by querying the map component for the next unexplored map area. We distinguished between explored areas, which are map areas where we know the weights of all vertex edges, and unexplored areas where we do not know the weights of all vertex edges.
If agents of the third group came to a vertex that had unknown edge weights to at least one neighbour vertex, they used the survey action to get the information as percepts.
These weights were then passed to the map component and they repeated the exploring, by querying for the next not surveyed vertex and going there.
To prevent multiple agents from going to the same vertex and exploring the same area, the map component used an internal locking mechanism.
For zoning we were not interested in a full coverage of the map.
This was because a full map exploration would have consumed a lot more steps while bringing only little improvement to the knowledge about it as a whole.
Of course it could be that at some point during the simulation we gain a full coverage of the map, but it was not a criteria to switch to the second phase in the simulation.

At some point in the simulation, around step 100, there were more agents available for exploring than vertices that needed to be surveyed. Responsible for this could be an almost fully explored map or a bottleneck vertex which is the only connection to other parts of the map.
Agents without an assigned vertex would be idle and waiting for the next step.
As we assumed that our agents would always be evenly placed on the map at the beginning of the simulation, we did not pursue a solution for the bottleneck vertex.
Instead, we decided to remove every agent from the exploring agent team which could not be assigned an unsurveyed vertex and put it in the zoning team.
In the zoning phase, all agents which were finished with their exploration phase were used to build zones.
This included the Explorer agents but excluded the Saboteur agents, because they were following their own aggressive strategy.
A detailed description of the zoning phase is given in \autoref{alg:zoning}.


%%%%%%%%%%%%%%%%%%
\subsection{Agent Specific Strategies$^\dagger$}\label{alg:agentstrategies}
Because each of the five different agent types (or \emph{roles}) in the MAPC scenario --- Explorers, Repairers, Saboteurs, Sentinels and Inspectors --- have different capabilities in terms of the actions they can perform, they must each act according to role-specific strategies and tactics in order for the team to perform well.
This section will give a short overview of how different agent types behave differently from each other.
\begin{description}
    \item[Explorer] agents are the only ones who can perform the \texttt{probe} action.
        Vertices must be probed in order to learn their value, which is critical for zoning.
        Accordingly, an Explorer will spend most of his time seeking out, moving towards and finally probing vertices whose value is not yet known.
        Since there are 6 Explorers in a team, care must be taken to make sure that multiple Explorers don't move towards the same unprobed vertex, as this is generally a suboptimal usage of their time.

        In our implementation, we consider all vertices in the map to be \enquote{worthy} of being probed and thus to know their value.
        However, due to the way our team performs zoning (see~\autoref{alg:zon_finding}), we prioritize vertices in a specific way which we call \enquote{cluster probing}.
        Because our zone calculation algorithm (see~\autoref{alg:zon_calculation}) puts agents around a centre vertex in a circular manner and with a maximum distance of two edges away, and because we want the Explorers' probing to help us with quickly finding and establishing high-value zones, Explorers should avoid probing vertices in e.g.\ a straight line.
        Rather, an Explorer's probing pattern should mimic the circular shape of the zone calculation algorithm.
        Because of this, for probing we prioritize unprobed vertices first by distance, then by the number of edges they share with already probed vertices.
        The result is that Explorer movement is similar to a spiral pattern (provided the Explorer isn't disturbed by e.g.\ nearby enemy agents).
        Explorer agents don't stop this probing pattern until they can no longer find any unprobed vertices.
    \item[Repairer] agents are the only agents who can perform the \texttt{repair} action for restoring health to disabled agents.
        Because the team loses out on possible points for every disabled agent in the team, to achieve a high score it is essential to quickly repair damaged agents.
        In our implementation, Repairers' actions are prioritized so that they will attempt to repair any disabled friendly agent in their visibility range, and it is the \enquote{job} of the disabled agents to find and move towards the closest friendly repairer.
        If a Repairer agent is aware of a friendly disabled agent outside of his visibility range, and the Repairer is currently not used for zoning, however, then the Repairer will also move towards the disabled agent.
    \item[Saboteur] agents are the only agents that can disable enemy agents using the \texttt{attack} action.
        In our implementation, a saboteur's role is very aggressively defined, and is prioritized thusly: if you see a non-disabled enemy, attack it; otherwise find and move towards an enemy you can attack.

        Throughout most of our development phase, Saboteur agents were the only agent type we would use the \texttt{buy} action for to extend their visibility range once for every Saboteur because we believed that this would give us an offensive edge against other teams.
        We decided to try out other buying strategies as well, however, by having our agents play matches against copies of themselves, except that the copies used different strategies for buying upgrades.
        Through this brief, empirical and rather informal testing period, we discovered that a surprisingly simple and novel strategy to buying upgrades actually led to persistently higher scores than our initial approach of buying one visibility range upgrade per Saboteur: by choosing a single Saboteur agent, which we call the \emph{uber-Saboteur}, and allowing him to buy an unlimited (well, limited only by the amount of money available) number of upgrades as needed, we were able to outperform teams using our more conventional approach of upgrade buying.
        To be more specific, our uber-Saboteur would buy an upgrade whenever no active enemy was within his visibility range and he had the money for it.
        The kind of upgrade (maximum energy, visibility range, maximum health or strength) depends on the relative improvement that buying that upgrade will bring to the upgrade-specific \enquote{module}.
        For example, if the uber-Saboteur has a maximum health of 3 and a visbility range of 1, then increasing the maximum health to 4 would be an improvement of \SI{33}{\percent}, while increasing the visibility range from 1 to 2 would be an improvement of \SI{100}{\percent} --- so the uber-Saboteur will choose to buy an upgrade to the latter.

        We also call in Saboteurs to defend a zone if it gets attacked by an enemy agent.
    \item[Sentinel] agents don't have a unique action that they can perform.
        Their strength is that they start with a visibility range of 3 by default, which is the highest of all the agent types.
        This is useful during exploration and to be warned of incoming enemy agents, but we don't use any Sentinel-specific logic in our implementation worth mentioning.
    \item[Inspector] agents are uniquely able to perform the \texttt{inspect} action.
        We consider it important to inspect every enemy agent once during each match to learn and store that agent's role (it is very important to know which enemy agents are Saboteurs, so that we can avoid them), but once that goal has been achieved, Inspector agents lose most of their importance.
        Achievement points for performing \texttt{inspect} actions are not awarded for inspecting an enemy agent more than once, and \texttt{inspect} is not needed to be able to tell if an agent is disabled (the \texttt{visibleEntity} percept includes the agent's current state).
        The only use case for inspecting an enemy agent more than once during a single match is to learn if they have bought any upgrades since the first time they were inspected.
        But because buying an upgrade for an agent is such a rare occurrence during the actual contest (cf.\ this year's and the last years' matches), we don't have to re-inspect very often --- we could probably just inspect each enemy agent once to learn their role and then leave it at that.
        In our implementation, however, we toggle an enemy agent to be ready to be inspected again 50 turns after it was last inspected.
\end{description}
 % Agent Specific Strategies

%%%%%%%%%%%%%%%%%%
%%%%%%%%%%%%%%%%%%
\subsection[Exploration]{Exploration$^\star$}\label{alg:exploration}
One precondition of the Agents on Mars Scenario is that all agents start with an empty belief base. Each agent does not know about its local and global environment. Of course every agent gets beliefs about its local environment very quickly by receiving percepts from the server. But the agent still does not know about the global environment. For our strategies it is crucial to have as much information about the overall environment as possible. Just think about finding and building the best global zones or pathfinding. So it is important to store somehow information about the map like vertices, edges between vertices, paths and agent positions. In \autoref{alg:map_cartographer} our initial approach with its down- and up-sides is described. After that a basic overview over the Distance-Vector-Algorithm and the impact on our map building approach is given in \autoref{alg:map_dv}. The chapter concludes with a description of the second approach we used and sticked to in \autoref{alg:map_javamap}


%%%%%%%%%%%%%%%%%%
\subsubsection[Cartographer Agent]{Cartographer Agent$^\star$}\label{alg:map_cartographer}
We decided very early in the development process that we do not want to store information which is needed by every agent in each single agent. The intention behind this decision was to reduce the effort in synchronizing and maintaining data between the single agents. Our initial approach was to install one omniscient pseudo agent we called the ``cartographer'' agent. The cartographer agent represented a map and had the only task to calculate shortest paths between given vertices and to store vertex and edge information like traversing costs, edges between vertices and vertex score points. Every agent told the cartographer agent about its environment related beliefs and the cartographer agent stored these beliefs. If an agent needed to know a shortest path, it just queried the cartographer agent and got the shortest path as an answer. Or if an agent needs to know if an vertex was already probed or surveyed, it just queried the cartographer. Shortly after implementing this approach, we encountered two major problems, which both resulted in serious performance issues. One problem was that pathfinding, which was done with the help of the Dijkstra-Algorithm, was executed every time an agent asked for a shortest path. This led to a lot of redundant queries and processing in the cartographer agent. The second problem was related to communication between agents. To understand the latter problem, one need to know that Jason uses a message box system for communication between agents. This means that every message a sender sends to a receiver is queued in the receivers message inbox. In every Jason lifecycle only one message is processed. Although a Jason lifecycle is a lot shorter than a server lifecycle, still after some execution time the inbox of the cartographer agent was so full, that the processing of messages lagged far behind the receiving of these messages. Both issues resulted in blocked agents, which had been waiting for the response of their queries for rounds.


%%%%%%%%%%%%%%%%%%
\subsubsection[Distance-Vector-Algorithm]{Distance-Vector-Algorithm$^\star$}\label{alg:map_dv}
What is it? How is it used in our context? What are advantages we gain from it? What is problematic (speed loss)?

%%%%%%%%%%%%%%%%%%
\subsubsection[JavaMap]{JavaMap$^\star$}\label{alg:map_javamap} %DSDV, Exploration and Map Generation

%%%%%%%%%%%%%%%%%%
\subsection[Repairing]{Repairing$^{\diamond,\dagger}$}\label{alg:repairing}
As was already mentioned in the scenario description, any agent can become disabled after being attacked by an enemy Saboteur agent.
To become disabled in the scenario means to lose all of the agent's health points.
Naturally, we have implemented several supporting strategies for avoiding enemy Saboteur agents when possible and parry when there is a Saboteur agent nearby.
However, following these strategies does not guarantee that the agent will never become disabled, mostly because an \enquote{escape route} that would put an agent out of the attack range of an enemy Saboteur agent does not always exist and not all agents are able to perform the \texttt{parry} action.
When an agent becomes disabled, it loses most of its functionality: only the \texttt{skip}, \texttt{recharge} and \texttt{goto} actions can be performed.
Repairer agents can also perform the \texttt{repair} action when disabled, although repairing costs more energy to perform when disabled.
Disabled agents also do not count towards establishing team ownership of a map vertex, and thus do not contribute to zone scoring.

In~\autoref{alg:agentstrategies} it was said that the primary task of Repairer agents is to repair others, and that they should prioritize the \texttt{repair} action whenever they see a disabled friendly agent within their visibility range.
It is important that disabled agents are repaired quickly, and for this to happen the disabled agent needs to be brought within repairing distance of a Repairer agent.
In our implementation, every time an agent becomes disabled, it receives the high priority goal \emph{getRepaired}.
Following the plan of this goal, an agent requests an available Repairer agent and its position from the \emph{MapAgent}.
If the returned Repairer agent position is the same the disabled agent's position, then the disabled agent only recharges and waits to get repaired.
Otherwise, the disabled agent simply moves towards the returned Repairer agent position.
If there is no Repairer agent available within the part of the map that the disabled agent knows how to reach, i.e.\ the explored subgraph the disabled agent is currently on, the disabled agent will instead attempt to expand the knowledge of the map by moving towards the closest unexplored vertex.

Assignment of agents to their Repairer agents is done inside our Java MapAgent.
To be more flexible, we decided to perform these assignments on every step.
This allows the system to adapt to constantly changing situations, such as when agents are moving, some other agent becomes disabled or a previously disabled agents is repaired, freeing up a previously agent Repairer agent.
The assignment itself is done based on the hop distances between agents.
First, all the distances between all disabled agents and all Repairer agents are calculated.
Then, the paths with the shortest distance are selected and the agents belonging to these paths are assigned to each other.
If all Repairer agents are assigned and there are still some unassigned disabled agents, they get assigned to the closest Repairer agent, even though that Repairer agent is already assigned to another disabled agent.
his assigning approach in most of the cases led to fast and effective repairing.

In addition to disabled agents moving towards their assigned Repairer agents, Repairer agents can also move towards their assigned disabled agents.
This behaviour is only possible during the exploration phase of the simulation, because in zoning mode Repairer agents moving will often to lead to the zone they help maintain being broken up, which we would like to avoid.
We implemented this by making Repairer agents explore the map in the direction of their assigned disabled agents, i.e.\ if the vertex the Repairer currently occupies on the way towards its assigned disabled agent is not surveyed, they survey it, otherwise they will continue on their path towards their assigned disabled agent.

What happens if a Repairer agent becomes disabled?
We decided to not treat this as a special case, but instead use the standard procedure of disabled agent to Repairer agent assignment.
For this to work, for Repairer agents the goal of repairing a disabled agent is higher on their priority list than waiting to get repaired themselves.
This helps to use all the repairers more effective and prevents the situation when all the Repairer agents are disabled and waiting to get repaired.

\subsection[Zone Forming]{Zone Forming$^\circ$}\label{alg:zoning}
% TODO: sometimes, agents ``must'', ``will'', ``have to'' and other times they simple (actively) ``do'' things.
Zone forming is the most important part in the \mars{}~\cite{ahlbrecht_mapc_2014}.% p.3
It describes the process of agents finding and occupying vertices in a way that they enclose a subgraph.
We called this process zoning.
For our approach, zoning takes place after the map exploration phase.
This should ensure that enough information about the graph has been gathered to calculate high valuable zones close to the agents' current positions.
The algorithm for calculating zones and determining which agents have to occupy which vertices is presented in \autoref{alg:zon_calculation}.
Said algorithm is used in the process of finding and negotiating a zone to build, which is described in \autoref{alg:zon_finding}.
After a zone that can be built has been found, agents are assigned dedicated roles.
These roles determine the agents' duties and tasks throughout the lifecycle of a zone which they are part of.
The lifecycle of a zone includes its creation, defence and destruction.
Both roles and the lifecycle are featured in the last \autoref{alg:zon_roles}.

%\newtheorem{definition}{Definition}
\subsubsection{Zone Calculation}
\label{alg:zon_calculation}
The graph colouring algorithm used by the MAPC server to determine occupied zones is described in detail in the MAPC 2014 scenario description~\cite{ahlbrecht_mapc_2014} and will not be explained again here.
Due to the way the server-side colouring algorithm works, placing $n$ agents on the map so that they establish the highest possible zone value per step is anything but straight-forward.
Even for $n = 1$, a single agent placed on an articulation point in the graph can establish a high-value zone if there are no enemy agents in either subgraph that it splits the map into.
\autoref{fig:articulation_points} shows an example.
\begin{figure}
  \centering
  \includegraphics[height=.5\textheight]{images/articulation_points}
  \caption{By occupying an articulation point, a single agent can ostensibly establish a a high-scoring zone---provided that there are no enemy agents inside the subgraph that is split off from the main graph.}
  \label{fig:articulation_points}
\end{figure}
To position themselves in an optimally-scoring way, agents could run the same algorithm locally to calculate the agent placement that will lead to the highest total sum of zone scores in each step by trying every possible permutation.
However, the number of ways to place $n$ agents on $k$ vertices is $C \left (n+r-1,r-1\right )= \frac{\left(n+r-1 \right )!}{n!\left(r-1 \right )!}$, a number that increases rapidly with $n$ and $k$.
In particular, there are $C \left (28+600-1,600-1 \right ) =\num{3.7463887025070038e+48}$ ways to place 28 agents on 600 vertices, which were the numbers used in the 2014 competition---far too many to calculate in real-time.
Finding an algorithm that calculates high-scoring zones in a limited computation time is one of the major challenges of the MAPC competition.
Our team developed a heuristic algorithm to calculating zones that will be explained below.
The goal is to find for every vertex in the graph a placement of agents around that vertex such that:
\begin{itemize}
  \item All of the centre vertex's one-hop neighbours (those vertices directly connected to the center vertex through a single edge) will be included in the zone.
  \item Agents can only be placed on the centre vertex's two-hop neighbours, which are those vertices that are be connected to the centre vertex through a minimum and maximum of two edges.
  \item The constructed zone's value per agent should be high.
        Ideally, it would be maximal, but the heuristic we use doesn't guarantee this.
\end{itemize}
\autoref{fig:zones} shows some examples of zones that are found using our heuristic algorithm.
\begin{figure}
  \centering
  \subcaptionbox{This zone was calculated for a centre vertex that only has a degree of 1, i.e.\ that is a leaf vertex.
                 Generally, it is preferable to place an agent on the cut vertex that leads to a leaf vertex rather than the leaf vertex itself, as this would establish at least a zone of equal size, and possibly larger.
                 \label{fig:zones_1}}[.49\linewidth]{\includegraphics[width=.49\linewidth]{images/zone1.png}}
  \subcaptionbox{Here, the centre vertex has a degree of 3, and the calculated zone remains compact with only two additional agents used.
                 A lot of optional agent positions remain.
                 \label{fig:zones_2}}[.49\linewidth]{\includegraphics[width=.49\linewidth]{images/zone2.png}}
  \\
  \subcaptionbox{A zone where the centre vertex has a degree of 5, and the zone uses a total of 4 agents.
                 \label{fig:zones_3}}[.49\linewidth]{\includegraphics[width=.49\linewidth]{images/zone3.png}}
  \subcaptionbox{A zone where the centre vertex has a degree of 7, and the zone uses a total of 9 agents.
                 \label{fig:zones_4}}[.49\linewidth]{\includegraphics[width=.49\linewidth]{images/zone4.png}}
  \caption{Four examples of zones calculated by the heuristic algorithm described in \autoref{alg:zon_calculation}.
           The green squares and triangles represent the placement of agents, where the triangle is the agent on the center vertex.
           Vertices marked with a small green circle are optional agent positions that can be used to expand the zone if there are agents left over at the end of the zone building, as described in \autoref{alg:zon_finding}.
           The green-colored area represents the zone that is established by the given agent placement.}
  \label{fig:zones}
\end{figure}
Every vertex in the graph is represented by a Java Vertex object, and the calculated zone is stored as a field of that object.
The steps of the algorithm are best detailed graphically, as in \autoref{fig:coloring}.
\begin{definition}
  Let $V$ be the set of vertices and $E$ the set of edges that the system knows about.
  For any $v \in V$, which we will use to denote the vertex that a zone is centered on, let $V_v^1 \subseteq V$ be the set of one-hop neighbours of $v$, that is, the set of vertices that share and edge with $v$: $V_v^1= \left\{w \middle|\left(v,w \right ) \in E\right\}$.
  Similarly, $V_v^2$ denotes the set of two-hop neighbours of $v$, i.e.\ the set of vertices that includes exactly those vertices that share an edge with any vertex in $V_v^1$, excluding those in $V_v^1$ and $v$ itself: $V_v^2= \left\{u \middle|\left(v,w \right ) \in E, \left(w,u \right ) \in E, u \notin V_v^1\cup\{v\}\right\}$.
  Let $V_v^{2+}$ be the entire two-hop neighbourhood of $v$: $V_{v}^{2+} = \{v\} \cup V_v^1 \cup V_v^2$.
  Additionally, let $A_v$ be an initially empty set that we will use to remember vertices we want to place agents on.
  A zone and its zone value are defined as specified by the graph colouring algorithm in~\cite{ahlbrecht_mapc_2014}.
  Then, the goal of the zone calculation algorithm is to find, for every $v$ in the graph, a set of agent positions $A_v \subseteq V_{v}^{2+}$ that establish a zone around $v$ so that the zone's value \emph{per agent} is high according to the heuristic used by the algorithm.
\end{definition}
Note that although $V_v^1$ and $V_v^2$ start off as defined above, by abuse of notation we will remove vertices from those sets as the algorithm progresses.
This does not mean that the structure of the graph has changed.
The algorithm for zone calculation is (re)-triggered every time a vertex in the vertex' two-hop neighbourhood (so within the ambit of the zone we're trying to calculate) is discovered during map exploration or changes its known value when it is probed by an Explorer agent, as these are the events that can lead to the position of agents that construct the zone changing.
The zone centered around vertex $v$ is is calculated through several steps:
\begin{enumerate}
  \item Initially, $A = \emptyset$, and $V_v^1$, $V_v^2$ and $V_v^{2+}$ as defined above.
        Iterate through every $w \in V_v^2$ and, for every $w$ that is connected to 2 or more vertices in $V_v^1$, $w \in V_v^2, \left(w, u_1 \right )\in E, \left(w, u_2 \right ) \in E, u_1 \neq u_2, u_1 \in V_v^1, u_2 \in V_v^1$, set $A := A \cup \left\{w\right\}$ and $V_v^2 := V_v^2 \setminus \left\{w \right \}$.
  \item For every $w \in V_v^2$, if $w$ is connected either directly or through a single one-hop neighbour of $v$ to any $u \in A$, remove it from $V_v^2$: $\forall w \in V_v^2: \exists u \in A: \left(\exists \left(w, u \right ) \in E \right ) \vee \left(\exists \left(w, x\right)\in E \wedge \exists \left(x, u\right) \in E \wedge \exists (x, v) \in E ) \right ) \rightarrow V_v^2 := V_v^2 \setminus \{w\}$.
      The reasoning behind this is that those vertices in $V_v^1$ that are neighbours of those in $A$ will already definitely be included in the zone, and the vertices we remove this way will not contribute towards our goal of including all one-hop neighbours $V_v^1$ in the zone for $v$.
  \item In the next step, \enquote{bridges} are discovered in the list of remaining two-hop neighbours $V_v^2$.
        A bridge is considered to be a connected triple of vertices where one of the vertices is directly connected to the other two.
        If such a bridge exists in $V_v^2$, all three involved vertices can be included in the zone around $v$ by placing an agent on either end of the of the bridge and leaving out the in-between vertex: $\forall w_1, w_2, w_3 \in V_v^2: \left(w_1, w_2\right ), \left(w_2, w_3 \right ) \in E \rightarrow A := A \cup \left\{w_1, w_3 \right \}, V_v^2 := V_v^2 \setminus \left\{w_1,w_2,w_3\right\}$.
        Since three vertices can be captured in the zone for the \enquote{cost} of two agents, we consider this a good exchange to make.
  \item Add $v$ to the list of agent positions: $A := A \cup \left\{v\right\}$.
  \item Next, the algorithm checks if all one-hop neighbours $V_v^1$ are connected to the agent positions $A$.
        This is frequently the case, but not always.
        If a remaining, unconnected one-hop $u \in V_v^1$ is found, we check if it is connected to one or more of the remaining two-hop vertices in $V_v^2$.
        If that is the case, we choose the neighbouring two-hop $w \in V_v^2$ with the highest vertex value and add it to the list of agent positions $A$.
        If no such two-hop vertex is found, we add the unconnected one-hop vertex to the list of agent positions---this is the only case where a one-hop vertex can be added to the list of agent positions:$\forall w \in V_v^1: \lnot \exists u \in A: \left(w, u \right ) \in E \rightarrow \left\{\begin{array}{lll}
A := A \cup \left\{x\right\}, V_v^2 := V_v^2 \setminus \left\{x\right\} & \textup{if} & \exists x \in V_v^2: \left(x, w \right ) \in E\\
A := A \cup \left\{w\right\} &\textup{else} &
\end{array}\right.$.
  \item Finally, we include the center vertex $v$ in the list of agent positions: $A := A \cup \left\{v\right\}$.
        Any vertices that remain in $V_v^2$ are saved as additional agent positions that could be used to extend the zone by otherwise idle agents, but unlike the vertices in $A$ vertices are not required to establish the initial smallest zone that we calculated.
\end{enumerate}
\begin{figure}
  \centering
  \subcaptionbox{Before zone calculation.
                 \label{fig:coloring1}}[.49\linewidth]{\includegraphics[width=.49\linewidth]{images/coloring1.png}}
  \subcaptionbox{After finding multiply-connected vertices in $V_v^2$ and removing their neighbouring two-hops.
                 \label{fig:coloring2}}[.49\linewidth]{\includegraphics[width=.49\linewidth]{images/coloring2.png}}
  \\
  \subcaptionbox{Finding bridges.
                 \label{fig:coloring3}}[.49\linewidth]{\includegraphics[width=.49\linewidth]{images/coloring3.png}}
  \subcaptionbox{After zone calculation.
                 \label{fig:coloring4}}[.49\linewidth]{\includegraphics[width=.49\linewidth]{images/coloring4.png}}
  \caption{The zone calculation algorithm shown in four steps.
           Vertices are coloured differently according to their current state as the algorithm progresses.
           Green vertices are vertices that an agent must be placed on, so those in $A$.
           Red vertices are those where placing an agent would be redundant because it does not extend the zone.
           Yellow vertices denote notes where agents could be placed to extend the zone, but are not considered optimal agent positions in the eyes of the algorithm.
           Numbers shown next to vertices represent their edge distance from the centre vertex.}
  \label{fig:coloring}
\end{figure}
While we consider our algorithm to find zones of acceptably high zone values per agent, it can easily be shown to be suboptimal.
For one, it only considers vertices within the two-hop neighbourhood of the centre vertex, and it is not difficult to think of possible graph structures where a different agent placement would lead to a better zone.
For example, if one of the remaining yellow vertices in \autoref{fig:coloring4} were an articulation point whose inclusion in the list of agent positions would add more than that single vertex to the zone, this would probably be a good vertex to place an agent on, yet this would not be discovered by our heuristic algorithm.

\subsubsection[Zone Finding Process.]{Zone Finding Process.$^\circ$}\label{alg:zon_finding}
This section explains how agents decide on what zones should be built.
Each zone finding process can end successfully or fail for any individual agent.
If it failed, the agent is not going to be a part of a zone and a new zone finding process is started.
The first part of this section covers what changes are made so that with every failed zone finding process a successful one becomes likelier.
If a zone finding process ended successfully, the most valuable zone known to the agents will be built.
This is ensured through agent communication which is presented in the last part of this section.
% It focusses on the communication between the agents.
Agents start looking for zones when they have finished the exploration phase.
As explained in \autoref{alg:exploration}, explorer agents do not only survey but probe as well.
Hence, other agents may finish the exploration phase earlier.
Furthermore, zones can be broken up at any time forcing the agents to start looking for a new zone again.
As a result, the zone finding process is in fact asynchronous.
Problems arising from this are mainly dealt with throughout the actual building of zones which is illustrated in \autoref{alg:zon_roles}.

In the beginning, all agents have to centrally register themselves when they are ready for zone building to indicate this availability.
Next, each agent uses the algorithm presented in \autoref{alg:zon_calculation} to determine the best zone in its neighbourhood.
The algorithm will only return zones which need at most as many agents to be built as there are registered agents.
This is to ensure that agents will not try to build zones for which there are not sufficiently many agents available.
The algorithm further uses a range parameter $k$.
It indicates the $k$-hop-neighbourhood up to which the algorithm will look for the best not yet built zone.
This range starts at $1$ and is incremented every time the agent finishes a zone finding process without being part of a zone afterwards.
As a result, it is more probable for an agent to find a zone with a high per agent score which has not been build yet.
Thus, it is also likelier for the agent to be part of a zone, because throughout every zone finding process only the most valuable zone is going to be built.
The range has a maximum to ensure that an agent will not look for zones too far away from it.
When a zone is broken up, the range will be reset, which is covered by \autoref{alg:zon_roles}.

After every agent interested in building a zone has determined the best zone in its neighbourhood, all such agents must send their best zone to all other agents.
This is because all agents ready to build a zone should know about and hence only try to build the best globally, not yet built zone.
At any time, every agent may only know about one zone.
This zone will be the best zone an agent is aware of at the moment.
Zones are being compared by their per agent score.
A higher score indicates a better zone.
Before building any zone, the agents will have to wait until the information about their best zone has reached all other agents.
This is ensured by the agent having to wait for all other agents to reply to him.
Therefore, when an agent receives information about a zone, it has two options.
One is to reply with a simple acknowledgement message expressing that it had received the message.
The other is to reply with its own zone in case that its zone is better.
Agents may not reply with information about a better zone if it is not their own.
This is to prevent duplicate messages.
Otherwise, multiple agents could reply with the same zone of which they had been informed about by the same agent.
Whenever an agent receives information about a better zone, it replaces its former knowledge about the best zone with the new one.
Agents which are not interested in building a zone but receive information about a zone simply ignore the message but reply with an acknowledgement.
This way, the sender will still be able to determine when every agent has processed the sent information.
In case the zone calculation algorithm did not return any zone to an agent, this agent has to ask all other agents for a zone.
It will accept the first reply containing zone information as its new best zone because it is better than no zone.
The agent will then continue similar to the earlier presented behaviour and wait until it received replies from all other agents.
After an agent has received all replies, it may start building a zone as illustrated in the next subsection. % TODO NB: this only works if the order is not changed again.

\subsubsection{Zone Building Roles and the Lifecycle of a Zone.}\label{alg:zon_roles}
This subsection describes the two roles exclusive to zone building. It covers the roles' associated tasks and duties throughout the lifecycle of a zone as well as the lifecycle itself. These roles are those of a \emph{coach} and a \emph{minion}. Each zone is built by one coach and a varying amount of minions. Minions are agents which are dedicated to build a zone by obeying their coach's orders. Every agent may only be part of one zone at a time. % The last sentence is probably redundant.
The roles are assigned when the zone finding process has ended and a concrete zone is about to be built. Agents keep either of these roles until the zone is broken up or they have to leave it. The roles regulate the agents' behaviour throughout the time they spend in a zone.
% This subsection describes the communication hierarchy during active zoning until its breakup

Before looking at border cases, an ideal case of a zone lifecycle is presented. There, the zone finding process described in \autoref{alg:zon_finding} ends with all agents knowing about the same best zone. This zone was found by one agent which will then become the zone's coach. Next, the coach informs the agents which will be part of the zone where to go to. On receipt of this message, the agents become minions and move to their designated node. The coach will also have to move to its node, which happens to be the centre node of the zone. Furthermore, the coach will unregister itself and all its minions to indicate their unavailability to build any other zone.
In a zone, minions serve no other purpose than to occupy their designated node. If a minion becomes disabled, it has to move towards a repairer agents. Due to this, it has to leave its node. Therefore, the zone can no longer exist in its original form. In such a case, a minion has to inform its coach about its departure. The coach must then tell all its other minions that the zone can no longer be maintained. Consequently, all affected agents drop their role and restart looking for zones as illustrated in \autoref{alg:zon_finding}.

In reality, the zone finding process is asynchronous. Therefore, it is likely that some agents start looking for a zone when others have nearly finished. As a result, there can be multiple groups of agents with different knowledge about which zone would currently bring the highest score per agent. Each group could then be expecting a different agent to become a coach. This interferes with the assumptions that each agent may only be in one zone and have only one role at a time. To solve this problem, coaches do not only inform their minions about where to move to. Instead, they also transmit the per agent score of the zone they want to build together with this agent. Any agent can then compare the received zone score with the zone it wanted to build before. If it is higher, it must inform the coach of its former zone or its minions if it had been the coach itself. In case that the proposed zone's score is lower than the zone the agent intended to form, it must inform the coach who just proposed the new zone. Said coach will then have to inform all its minions that its zone is not going to be built.

Besides coaches and minions, there are also other agents who might be looking for a zone but will not be part of the one which will be built. Such an agent will have to start a new zone finding process. Prior to that though, it will look for any highly valuable node in its surrounding which is not yet occupied by anyone and move there. The range to look for such a node is the same as the range for finding a zone in the agent's neighbourhood presented in \autoref{alg:zon_calculation}. It is increased after every zone finding process which does not result in a zone where the agent is part of. The idea is that with a wider range, the probability to find a highly valuable zone increases. Additionally, the agent will likelier move farther away from its position in case it is not part of the zone to be built. This should further ensure that zones are only proposed multiple times as best zones if they have a very high per agent score.

We assume due to our zone calculation algorithm that a node within a zone will be occupied by at most one agent. %TODO it won't be called a colouring algorithm later on.
Then, any enemy agent close to a zone endangers it. This is because a zone may not spread across an enemy inside of it~\cite{ahlbrecht_mapc_2014}. % p.12
Moreover, enemy saboteur agents can disable agents inside a zone, which similarly destroys the zone in its original form~\cite{ahlbrecht_mapc_2014}. % p.11
Hence, coaches check once per step whether an enemy agent is close to the zone. If this is the case, the coach broadcasts a message to all saboteur agents to come and defend the zone. Saboteur agents which are not already defending a zone bid for this. The saboteur agent closest to the zone's centre will win the bidding. It then moves towards the enemy to disable it. If the coach detects in a next step that the enemy moved away from the zone, it will cancel the zone defence. The coach does so by using another broadcast as it does not know which saboteur agent was selected to defend the zone.

Explorer agents will still be probing when the first agents start looking for zones. Therefore, the most valuable zones may change with more and more nodes being probed. To prevent that agents build a zone once and stay there for ever if no agents attack them, zones will be split up periodically. The periodic trigger is linked to the overall steps of the simulation and not the lifetime of each zone respectively. Consequently, agents from different zones will have to restart looking for a zone at the same time. In addition to allowing new zones to be build which take the information of the newly probed nodes into account, this also allows for agents to start the zone finding process in a less asynchronous fashion.
% TODO: proper ending of this section

