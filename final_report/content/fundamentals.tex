\subsection{MAPC: Contest and Scenario}
\todo{Write this section}
What is the general scenario? Agents on mars trying to find water in a competitive manner against another team. Explain the concept of zones, gaining points, achievements and also how agents differ from each other. Introduce the notion of \emph{zoning} as the process of finding and forming a zone (including upkeeping by defense?).
I've put this section first so the later chapters can already rely on the reader knowing what the scenario is about. Furthermore, we can then directly rule out concepts which we are presenting by applying them theoretically onto the scenario/our needs.

\subsection{Agent Programming Concepts}
\todo{Write this section}
\subsubsection[BDI.]{BDI$^\blacktriangle$}
\subsubsection[BDI]{BDI$^{\blacktriangle,\diamond}$}\label{fun:BDI}
Using cognitive modelling techniques for agent development enable autonomous behaviour and reasoning for automated problem solving.
This also allows the analysis of real world behaviour through simulations with agents.
Problems can hence be automatically solved through self-organised groups of agents and new knowledge can be drawn from such simulations.
The beliefs, desires and intentions (short: \emph{BDI}) model, is a widespread software model for developing intelligent agents situated in complex and dynamic environments.
There, the basic characteristics of an agents' mental state are expressed through beliefs, desires and intentions which are explained in more detail later in this section.
The BDI logic system is quite easy to implement in software agents and to some extent promises human-like reasoning behaviour.
Hence, it has been widely used in the field of artificial intelligence in computer science.
This section begins with the roots of the BDI model by mentioning the scientific papers that led to it.
It then continues with an explanation of practical reasoning after which the components of the BDI model are presented in greater detail.
% TODO if there is more to come, I will add it here. I still haven't made it through this text.

In 1987, Bratman~\cite{MICHAEL_PlansResource_1988} discussed the relationship between beliefs, desires, intentions and actions as well their roles in agent behaviours.
This can be seen as the introduction of the BDI model which was revised in 1991 by Rao and Georgeff~\cite{rao_modeling_1991}.
They formalised the model to a first order logic and treated beliefs, desires and intentions as three modal operators while giving intentions equal importance compared to beliefs and desires.
Rao and Georgeff then continued to apply this theoretical foundation to concrete BDI agents, applying them in an airline traffic management application~\cite{Rao_BDITheory_1995}.
Nowadays, BDI agents are deployed in high technology industrial areas such as fault diagnosis on space shuttles, but also in commercial fields and entertainment such as robot soccer games.

The BDI model has its roots in philosophy as found in Bratman's theory of practical reasoning~\cite{Sebastian_Hierarchical_2006}.
Practical reasoning involves two important processes: deciding what goals should be achieved, and how they should be achieved.
The former process is known as \emph{deliberation}, the latter as \emph{means-ends reasoning}~\cite{Gerhard_MultiSystem_1999}.
Means-ends reasoning is a method to make plans for achieving goals based on current states.
The idea of means-ends reasoning is to reduce the difference between the current state and a goal, furthermore, the method is applied recursively.
When an agent is placed in an environment, it should autonomously decide what to do and how to do it.
In general, an agent could execute many different actions but maybe only a few of them would have a desired effect on the environment.
Various external properties can have influence on the feasibility of achieving the agent's goals\todo{\textbf{@yuansun1990}: this part confused me a bit. I don't think, \enquote{affairs} was the right word here. Please check if I still understood what you wanted to say.}.
The deliberation process filters what options are actually possible in the current state.
It then determines which of these options will become intentions.
For example, if you are thirsty and standing in a supermarket, then you might be faced with the decision to choose a drink.
There can be a lot of options like wine, beer, water, lemonade or juice.
However, picking up a bottle of wine or beer is not allowed in Germany for people under the age of 16. % Germany, sadly a proud nation of drunkards.
After collecting all the available options, you must choose and commit to some of them which become intentions next.
Subsequently, we need the mean-ends reasoning process to plan how to achieve these intentions.
If your intention is to buy a bottle of water, you may then plan to go to the shelf with water on it, reach for such a bottle, take it to the checkout counter and pay for it.
Finally, you have to execute this plan to buy the bottle of water.

The BDI model, as an applicable theory of practical reasoning, consists of three components which are beliefs, desires and intentions.
These are subsequently explained each.

Beliefs represent the informational state of the agent~\cite{Rao_BDITheory_1995} and are updated appropriately after each sensing action.
They may be implemented as a variable, a database, a set of logical expressions or some other data structure~\cite{Rao_BDITheory_1995}.
Beliefs model the agent's look at the world.
They can express information about the environment, other agents or the agent itself.
An agent may update its beliefs at any time.
Agents receive new information from the perception of the environment and the execution of intentions.
An agent can use sensors to perceive the environment and store their output as beliefs.
Beliefs are not the same concept as knowledge.
Knowledge is the realisation of a fact whereas a belief models \enquote{knowledge} which is believed by the agent.
Beliefs do not necessarily have to be true from a global perspective.
But from the agent's perspective, they are.
Hence, beliefs can be seen as local knowledge.
Likewise, beliefs can be global knowledge when they are true in a global sense.

Desires represent the motivational state of the agent~\cite{Rao_BDITheory_1995}.
In other words, desires represent objectives or situations that the agent would like to accomplish or bring about.
Desires can be but do not need to be achieved.
% Although similar, desires can be distinguished from goals.
Multiple desires can be inconsistent with each other and the agent does not need to know the means of achieving these desires.
They are inputs to the agent's deliberation process, which results in the agent choosing a subset of desires that are both consistent and achievable.
Such consistent and achievable desires are usually called \emph{goals}~\cite{Gerhard_MultiSystem_1999}.
For example, sleeping and working may be both one's desires, but they can not both be one's goals at the same time because they conflict with each other.

Intentions are desires or actions that an agent committed itself to achieve~\cite{Alejandro_LearnBDI_2004}.
Their meaning is stronger than that of desires.
Desires are merely wishes that may be achieved or may be not, but intentions are decided to be achieved to a reasonable extent.
Michael Wooldridge~\cite{Gerhard_MultiSystem_1999} concluded four functions of intentions in practical reasoning:
\begin{itemize}
  \item Intentions driving means-ends reasoning means that intentions have decisive influences on the actions which the agent will execute.
    Agents are expected to determine ways of achieving intentions.
  \item Intentions constrain future deliberation.
    This means that options which would be in conflict with an already chosen intention will not be considered.
  \item Intentions persist means that intentions will not be given up unless there is a rational reason to do so.
    This is important because if an agent immediately dropped its intentions without devoting resources to achieving them, it will never achieve anything~\cite{Gerhard_MultiSystem_1999}.
    Hence, intentions are committed desires which can not be easily abandoned.
    A reason to drop an intention nevertheless could be e.g. when an external event made the intention impossible or a lot more difficult to achieve.
  \item Intentions influence beliefs upon which future practical reasoning is based.
    This expresses that an intention is believed to be eventually achieved under \enquote{normal circumstances}.
    Practically, after deciding on an intention, the agent will add the intention as a belief.
    This allows the agent to plan for the future on the assumption that the intention will be achieved.
\end{itemize}
Intentions play an important role in the BDI model as they lead to actions and influence future beliefs and intention selection.

Beliefs, desires and intentions are the foundation of BDI agents.
Yet, further components are needed to build the connection between these three components and thereby implement BDI agents.
Naturally, the architecture of BDI agents differs in detail depending on their tasks.
However, they all share a core architecture which is depicted in \autoref{fig:bdi_architecture}.
Its components are explained in the following paragraphs.
% TODO we will have to place this somewhere else as it contains elements like BRF which are explained waaay later.
\begin{figure}[htbp]
  \centering
  \includegraphics[width=\textwidth]{images/BDIAr}
  \caption{Brief BDI architecture~\cite{BDIA}}
  \label{fig:bdi_architecture}
\end{figure}

The sensors of a BDI agent perceive the environment and convert the perceptions to signals as inputs for the belief revision function (short: \emph{BRF}).
This function collects the external perceptions as well as the beliefs which are already stored in the beliefs.
It then processes the information and updates the beliefs accordingly.
The belief revision function prefers minimal change over modifying a lot and hence tries to preserve as much information as possible~\cite{Antje_SpatialBelief_2011}.
While processing the new and old information, the BRF will also solves inconsistencies, e.g.\ when an agent perceives information which contradicts a belief it has.

The belief set's data may be modelled as sentences, rules or some other manifestations.
In the \emph{AGM approach} (named after their proponents, Alchourrón, Gärdenfors, and Makinson)~\cite{alchourron_revision_1985}, an agent's beliefs are modeled by a deductively closed set of formulas called a \emph{beliefs set}~\cite{James_revise_2011}. 
This approach is broadly used in belief revision research.
It makes several postulates for one revision operator which mapping each belief set and one sentence of beliefs to generate new beliefs.
Meanwhile, these postulates allows the belief revision process to retain as many previous beliefs as possible to reduce the amount of change.

The option generator reads the beliefs information and returns a list of options, which are current desires, into the desires set.
It determines the desires depending on the agent's current beliefs and current intentions.
The desires set contains desires, which are the possible courses of actions available to the agent.
These desires can be achievable or not.
The filter determines the agent's intentions depending on current beliefs, desires, and intentions.
It needs to reason about more situations than the functions in previous steps.
Desires will become more rational after filtering.

The intentions set stores the agent's current focus - the actions, which are going to be executed or committed to be executed at some point of time.
Once an intention is adopted, it should not be immediately dropped out because of the commitment.
But in some situations the intentions should be given up.
There are three commitment strategies proposed in Rao and Georgeff's work: blind, single minded and open minded.
A blindly committed agent is an agent who maintains his intentions until he believes that he has achieved them.
A single minded committed agent is an agent who maintains his intentions as long as he believes that they are still options.
An open minded committed agent is an agent who maintains his intentions as long as they are still goals~\cite{Roberto_BDIATL_2005}.
The action selection function determines the actions to perform depending on current intentions.
We would achieve nothing if we just have intentions instead of knowing how to do it.
Normally, there is a plan library of mappings between the intentions and actions.
The intentions go through the planner until the actions which mapped to the corresponding intentions are found.
Finally, a plan of how to achieve the intentions come out and the agent will execute these actions.

For understanding the relationships between the seven main components of BDI agent architecture, one table~\footnote{This table summarizes the content of BDI model architecture which from Michael Wooldridge's work\cite{Gerhard_MultiSystem_1999}} is presented as follows:

\begin{table}[!hbp]
  \label{tab:BDIC}
  \begin{tabularx}{\textwidth}{|l|p{5cm}| >{$}X<{$} |}
  \hline
  \textbf{Component} & \textbf{Meaning} & \textbf{Formalisation} \\
    \hline
    Beliefs set & Information about the current environment which the agent has & B \\
    \hline
    Belief revision function & determines a new set of beliefs depending on perceptual inputs and the agent's current beliefs & B \times P \to B\\
    \hline
    Options & determines desires depending on the agent's current beliefs and current intentions & B \times I \to D \\
    \hline
    Desires set & possible courses of actions available to the agent & D \\
    \hline
    Filter & determines the agent's intentions depending on current beliefs, desires, and intentions & B \times I \times D \to I \\
    \hline
    Intentions set & the agent's current focus & I \\
    \hline
    Action selection function & determines an action to perform depending on current intentions & I \to A  \\
    \hline
  \end{tabularx}
  \caption{Components of brief BDI agent architecture}
\end{table}

\autoref{tab:BDIC} shows the order of using the seven components of BDI architecture as well as giving the formulas for each function.
If we denote $Bel$ a set of all possible beliefs, $Des$ a set of all possible desires and $Int$ a set of all possible intentions.
Then an agent's state can be presented as $(B,D,I)$ with $B \subseteq Bel, D \subseteq  Des, I \subseteq  Int$~\cite{Gerhard_MultiSystem_1999}.
$P$ is a set of current perception which are obtained by the sensors of the agent.
We can understand the process of BDI agent working better through seeing this table.
Firstly, $B$ stores some beliefs which are read by BRF (belief revision function) while it getting the perception from the sensors.
After operating BRF, some of beliefs are removed, some are added, some are modified and so on.
So the new beliefs set $B$ built on basis of $P$ and original $B$.
Subsequently, Options use the new $B$ and current intentions set $I$ to determine the desires set $D$ and store it.
Moreover, the filter select intentions by referencing $B,D,I$, then the new intentions set comes out.
Finally, the action selection function makes a plan to execute actions to achieve $I$.

The process of $B \times P \to B$, $B \times I \to D$ and $B \times I \times D \to I$ belongs to deliberation.
They are deliberated in-depth gradually and the range of intentions are narrowed, especially the filter which should consider of all there datasets.
At last, the intentions are limited in particular ranges.
The plans will be made more effective and more targeted.
$I \to A $ can be treated as the process of means-ends reasoning, whose output is planning.
$B,D I$ are connected to function parts instead of connecting to each other directly.
They are just databases and need rules or mechanism to help them execute actions.

With the increasing needs of intelligent agents, more and more applications base on BDI model are applied in our life.
For example, PRS~\cite{Ingrand_PRS_1992} and dMARS~\cite{Mark_dMARS_2004} are both BDI-based systems for the reaction control system of the NASA Space Shuttle Discovery.
Additionally, the air-traffic management system OASIS~\cite{Magnus_OASIS_1992} is well-known as a BDI-based agent.
The system architecture of OASIS is made up of one aircraft agent for each arriving aircraft and a number of global agents, including a sequencer, wind modeller coordinator and trajectory checker~\cite{Rao_BDITheory_1995}.
Furthermore, robot soccer which is designed using BDI model becomes very popular in universities.
We can feel that, BDI agents bring many profits to human beings, they make the life more convenient.
However, there is still space for development in this field.

Although the BDI model is developed during about 30 years, some obstacles are not overcome and some challenges are still there.
The goals of most BDI implementations are not explicit.
The agents should reason about the goals from the current beliefs and intentions.
Besides, the BDI model contains three attributes: beliefs, desires and intentions.
In some situations, not all the three attributes are needed.
Sometimes, an agent collects the beliefs and jumps to intentions directly without desires.
However, for some distributed multi-agents, just three attributes are not sufficient to execute the actions.
Furthermore, the agents in the multi-agents system do not have an explicit mechanism for interaction and integration among them.
When an increasing number of agents join the system, the interaction with each agent will be more and more difficult.
As an intelligent agent, the BDI agent do not have a good ability to learn from the past behavior or other agents’ behavior.
So that the rate of development will not be high, lacking mechanisms to learn from others.
However, BDI model has its own advantages.
Beliefs, desires and intentions are similar to the mental activities of human beings.
Therefore, it is not easy to construct the logics or mechanisms for it.
With the wildly used of computers and mobile devices, the situation of multi-agent interaction will be better.
As many computer languages and logic languages are grasped by more people, the BDI agent will bring human beings more surprises.

Beliefs, desires and intentions were introduced in this section.
The BDI agent belongs to the kind of intelligent agents which are autonomous, computational entities.
The BDI agent executes actions on the basis of BDI model that containing three main attributes which have close relationship between each other.
The brief BDI agent architecture is a clear description of process of how BDI agents work.
They follows the practical reasoning theory.
Different BDI implementations show different architectures, but the in core of these agents are still notions of beliefs, desires and intentions.

With an increasing number of BDI applications coming into the human life, more challenges come up too.
One of such challenges, which is the correctness of agents' behaviour, is described in the ~\autoref{fun:formal_methods} that follows.
Also, in the following section after introducing all the required logical operations and methods, a sample structure of an abstract BDI interpreter will be given, which also continues the topic discussed in this section.


\subsubsection[Formal Methods.]{Formal Methods$^\diamond$}
One of the challenges for multi-agent systems is how to make sure that the agent will not behave unacceptable or undesirable? Agents may act in complex production environments, where failure of a single agent may cause serious losses. Formal methods had been used in computer science as a basis to solve correctness challenges. They represent agents as a high level abstractions in complex systems. Such representation can lead to simpler techniques for design and development.

There are two roles of formal methods in distributed artificial intelligence that are often referred to. Firstly, with respect to precise specifications they help in debugging specifications and in validation of system implementations. Abstracting from specific implementation leads to better understanding of the design of the system being developed. Secondly, in the long run formal methods help in developing a clearer understanding of problems and their solutions. \cite{Singh_99}

To formalize the concepts of multi-agent systems different types of logics are used, such as propositional, modal, temporal and dynamic logics. In the following several paragraphs these logics, their properties and introduced operators will be briefly discussed. Describing the details of interpretations and models of each individual logic is not the purpose of this report and is left out for further reading.

Propositional logic is the simplest one and serves as a fundament for logics discussed further in this section. It is used for representing factual information and in our case is most suitable to model the agent's environment. Formulas in this logic language consist of atomic propositions (known facts about the world) and truth-functional connectives: $\land,\lor,\neg,\rightarrow$ which denote "and" "or" "not" and "implies" respectively. \cite{Enderton_72}

Modal logic extends propositional logic by introducing two different modes of truth: possibility and necessity. In the study of agents, it is used to give
meaning to concepts such as belief and knowledge. Syntactically modal operators in modal logic languages are defined as $\Diamond$  for possibility and
$\Box$ for necessity. The semantics of modal logics is traditionally given in terms of sets of the so-called possible worlds. A world here can be interpreted as a possible state of affairs or sequence of states of affairs (history). Different worlds can be related via a binary accessibility relation, which tells us which worlds are are within the realm of possibility from the standpoint of a given world. In the sense of the accessibility relation a condition is assumed possible if it is true somewhere in the realm of possibility and it is assumed necessary if it is true everywhere in the realm of possibility. \cite{Saul_63}

Dynamic logic is also can be referred to as modal logic of action. It adds differen atomic actions to the logic language. In our case atomic actions may be represented as actions that agents can perform directly. This makes dynamic logic very flexible and useful for distributed artificial intelligence systems. Necessity and possibility operators of dynamic logic are based upon the kinds of actions available. \cite{Kozen_90}

Temporal logic is the logic of time. There are several variations of this logic such as:
\begin{itemize}
  \item Linear or Branching: single course of history or multiple courses of history.
  \item Discrete or Dense: discrete steps(like natural numbers) or always having intermediate steps (like real numbers).
  \item Moment-based or Period-based: atoms of time are points or intervals.
\end{itemize}
We will concentrate on discrete moment-based models with linear past, but consider both linear and branching futures.

Linear temporal logic introduces several important operators. $p\cup q$ is true at a moment $t$ on a path, if and only if $q$ holds at a future moment on
the given path and $p$ holds on all moments between $t$ and the selected occurrence of $q$. $Fp$ means that $p$ holds sometimes in the future on the given path. $Gp$ means that $p$ always holds in the future on the given path. $Xp$ means that $p$ holds in the next moment. $Pq$ means that $q$ held in a past moment. \cite{Singh_99}
%
\begin{figure}[h!]
\caption{An example branching structure of time. (source: \cite{Singh_99})}
\centering
\includegraphics[width=0.6\textwidth]{images/branching_logic.png}
\label{sci:for_branching_figure}
\end{figure}

Branching temporal and action logic is built on top of both dynamic and linear temporal logics and captures the essential properties of actions and time that are of value in specifying agents. It also adds several specific branching-time operators. $A$ denotes "in all paths at the present moment". The present moment here is the moment at which a given formula is evaluated. $E$ denotes "in some path at the present moment". The reality operator $R$ denotes "in the real path at the present moment". Figure \ref{sci:for_branching_figure} illustrates the example of branching time for two interacting agents.

For modeling intelligent agents quite often used BDI concept, which was described earlier in this report. BDI stands for three cognitive specifications of agents: beliefs, desires, intensions. To model logic of these specifications we will need to introduce several modal operators: $Bel$ for beliefs, $Des$ for desires, $Int$ for intensions and $K_h$ for know how. Considering these operators, for example, the mental state of an agent who desires to win the lottery and intends to buy a lottery ticket sometime, but does not believe that he will ever win can be represented by the following formula: $DesAFwin \land IntEFbuy \land \neg BelAFwin$. For simplification in future we will consider only those desires which are mutually consistent. Such desires are usually called goals.

It is important to note several important properties of intensions, which should be maintained by all agents\cite{Singh_92}:
\begin{enumerate}
  \item Satisfiability: $xIntp\rightarrow EFp$. This means that if $p$ is intended by $x$, then it occurs eventually on some path. Intension following this condition is assumed satisfiable.
  \item Temporal consistency: $(xIntp \land xIntq)\rightarrow xInt(Fp \land Fq)$. This requires that if an agent intends $p$ and intends $q$, then it  (implicitly) intends achieving them in some undetermined temporal order: $p$ before $q$, $q$ before $p$, or both simultaneously.
  \item Persistence does not entail success: $EG((xIntp) \land \neg p)$ is satisfiable. This is quite intuitive: just because an agent persists with an intention does not mean that it will succeed.
  \item Persist while succeeding. This constraint requires that agents desist from revising their intentions as long as they are able to proceed properly.
\end{enumerate}

The introduced above concepts may be used in each of two roles of formal methods introduced earlier. There are two mostly used reasoning techniques to decide agent's actions: theorem proving and model checking. The first one is more complex in terms of calculations, when the second one is more practical, but it requires additional inputs, though it does not prove to be a problem in several cases.

Considering the practical implementation, the architecture of abstract BDI-interpreter can be described as follows. The inputs to the system are called events, and are received via an event queue. Events can be external or internal for the system. Based on its current state and input events the system selects and executes options, corresponding to some plans. The interpreter continually performs the following: determines available options, deliberates to commit some options, updates its state and executes chosen atomic actions, after that it updates the event queue and eliminates the options which already achieved or no longer possible.
%
\begin{lstlisting}
BDI-Interpreter
initialize_state();
do
    options := option-generator(event-queue, B, G, I);
    selected-options := deliberate(options, B, G, I);
    update-intentions(selected-options, I);
    execute(I);
    get-new-external-events();
    drop-successful-attitudes(B, G, I);
    drop-impossible-attitudes(B, G, I);
until quit.
\end{lstlisting}

As was mentioned above options are usually represented by plans. Plans consist of of the name or type, the body usually specified by a plan graph, invocation condition (triggering event), precondition specifying when it may be selected and add list with delete list, specifying which atomic propositions to be believed after successful plan execution. Intentions in this case may be represented as hierarchically related plans.

Getting back to the algorithm and assuming plans as options, the option generator may look like the following.
Given a set of trigger events from the event queue, the option generator iterates through the plan library and returns those plans whose invocation condition
matches the trigger event and whose preconditions are believed by the agent.
%
\begin{lstlisting}[mathescape]
option-generator(trigger-events, B, G ,I)
options := {};
for trigger-event $\in$ trigger-events do
    for plan $\in$ plan-library do
        if matches(invocation(plan, trigger-event) then
            if provable(precondition(plan), B) then
                options := options $\cup$ plan;
return options.
\end{lstlisting}

Deliberation of options should conform with the execution time constraints, therefor under certain circumstances random choice might be appropriate. Sometimes lengthy deliberation becomes possible by introducing metalevel plans into plan library, which form intentions towards some particular plans.
%
\begin{lstlisting}[mathescape]
deliberate(options)
if length(options) $\leq$ 1 then return options;
else metalevel-options :=
            option-generator(b-add(option-set(options)));
    selected-options := deliberate(metalevel-options);
    if null(selected-options) then
        return random-choice(options);
    else return selected-options.
\end{lstlisting}

Coordination is one of the core functionalities needed by multiagent systems. Especially when different agents autonomous and have different roles and possible actions.

One of the approaches developed by Singh \cite{Singh_97} represents each agent as a small skeleton, which includes only the events or
transitions made by the agent that are significant for coordination. The core of the architecture is the idea that agents should have limited knowledge about designs of other agents. This limited knowledge is called a significant events of the agent. Events can be of the four main types:
\begin{itemize}
  \item flexible, which can be delayed or omitted,
  \item inevitable, which can be only delayed,
  \item immediate, which agent willing to perform immediately,
  \item triggerable, which the agent performs based on external events.
\end{itemize}
These events are organized into skeletons that characterize the coordination behavior of agents. The coordination service is independent of the exact skeletons or events used by agents in a multiagent system.

To specify coordinations a variant of linear-time temporal language with some restrictions is used. For that purpose two temporal operators are introduced: $\cdot$ - before operator, and $\bigodot$ - the operator of concatenation of two time traces, first of which is finite. Such special logic allows a variety of different relationships to be captured. 

Overall, formal methods provide a logic abstraction for multiagent systems. They help to find self-consistent models of agent's behavior. However relatively high complexity do not allow these methods to be implemented in real time systems. Therefore the role of formal methods nowadays is limited to debug, validate and design purposes.

In our project we unfortunately did not apply any formal methods for debugging or validating, mostly because of the limited time for development.

\subsubsection[Negotiation and Argumentation.]{Negotiation and Argumentation$^\diamond$}
In multi-agent environment, where each agent has its own beliefs, desires and goals, achieving a common goal usually require some sort of cooperation. It most of the cases it can be achieved through communication and negotiation among groups of agents. Often negotiation is supported by some arguments which help to identify which agent is more suitable for completing certain task. Among them could be better position, better resources for completing the task, importance of current goal and so on. Some arguments can be also used to change the intentions of other agents. This could be the arguments like reserving the node to explore or the enemy to attack and many others. Argumentation is essential when agents don't have the full knowledge about other agents or environment. In such cases exchanging information helps to develop the consensus and make cooperative decisions.

To negotiate effectively a BDI agent requires the ability to represent and maintain the model of its own properties, such as beliefs, desires, intentions and goals, reason with other agents' properties and be able to influence other agent's properties \cite{Kraus_98}. These requirements should be supported by the agent programming language we choose for our project. 

As was mentioned above, negotiation is performed through communication. Negotiation messages can be of the following three types: a request, response, or a declaration. A response can take the form of an acceptance or a rejection. Messages can also have several parameters for justification or transmitting negotiation arguments. The arguments are produced independently by each agent using the predefined rules, which will be discussed later in this subchapter. Every agent can send and receive messages. Evaluating a received message is the vital part of negotiation procedure. Only the evaluation process following an argument may change the core agents' beliefs, desires, intentions or goals. 

There are always several ways of modelling agents for negotiation. Agents can be bounded if they do not believe in "false"; omniscient if their beliefs are closed under inferences; knowledgable if  their beliefs are correct; unforgetful if they never forget anything; memoryless if they do not have memory and they cannot reason about past events; non-observer if their beliefs may change only as a result of message evaluation; cooperative if they share the common goal \cite{Kraus_98}. For our project in most of the cases we assumed agent as knowledgable and memoryless - agents remember only about the current round of negotiation and abolish previous round results, when the new round starts. During the zone building process the agents also act as cooperative, since they share the common goal of building a zone.



\subsubsection{Agent Societies}
\todo{add, adapt and improve Rahul's part if it fits and is helpful for our later work}
\subsection[Agent Programming Languages]{Agent Programming Languages$^\circ$}
We investigated several agent programming languages, proposed by our supervisors, for their suitability for the \enquote{Agents on Mars} scenario.
Our goal was to determine which specialized language we wanted to use for multi-agent programming, if any.
The following sections present the basic structure of various languages together with examples.
These examples are unrelated to the \enquote{Agents on Mars} scenario and are kept simple for ease of understanding.
Using the Mars-scenario for examples instead would have meant to either make them complex or to trivialise them to a point where they become too superficial to suit the scenario.
\autoref{fun:apl_sitCalc} first introduces the situation calculus.
Although not an agent programming language, it serves as a foundation of the logic programming language GOLOG presented in \autoref{fun:apl_golog}.
It also helps in understanding the subsequent \autoref{fun:apl_flux} which summarises the main concepts of FLUX.
FLUX is another logic programming language which was partly motivated by the flaws of GOLOG.
\autoref{fun:apl_jadex} introduces a Java-based agent programming language. After that, AgentSpeak(L) is presented in \autoref{fun:apl_asl} which is another logic programming language.
Jason is an interpreter for this language and is discussed in \autoref{fun:apl_jason}.
The section focuses mainly on the extensions that Jason adds to AgentSpeak(L).
The final \autoref{fun:apl_choice} summarises the previous sections and explains our decision for choosing Jason.

\subsubsection[Situation Calculus]{Situation Calculus$^{\circ,\dagger}$}\label{fun:apl_sitCalc}
This section gives a short summary of the situation calculus, which was first introduced by McCarthy and Hayes~\cite{mccarthy_philosophical_1969}.
The situation calculus is mainly a first-order logic but also uses second order logic to encode a dynamic world~\cite{levesque_golog:_1997}. %60
It is a theoretical concept and is consequently not applicable to multi-agent scenarios without any concrete implementation.
Yet, it is being presented to serve as basis for the later illustrated languages GOLOG and FLUX.
The situation calculus consists of the three first-order terms: \emph{fluents}, \emph{actions} and \emph{situations}~\cite{mccarthy_philosophical_1969,boutilier_decision_2000}. %18+,356
Fluents model properties of the world.
Actions may change fluents and hence may modify the world.
Every action execution creates a new situation.
This is because a situation is a history of actions up to a certain point in time starting from the initial situation $s_0$~\cite{schiffel_reconciling_2006,levesque_golog:_1997}. %289, 60+
There can only be one initial situation as it models the situation before any action has been executed~\cite{pirri_contributions_1999}. %329

Fluents can be evaluated to return a result.
As they are situation dependent, the evaluation result may change over time.
Fluents are distinguished into \emph{relational fluents} and \emph{functional fluents}~\cite{levesque_golog:_1997}. %3
Relational fluents can hold in situations.
Their evaluation hence may return either true or false~\cite{boutilier_decision_2000}. %356
An example is given in \autoref{f_hasCoffee}.
It expresses whether or not the agent $p$ has a cup of coffee in situation $s$.
\begin{equation}\label{f_hasCoffee}
  \textit{hasCoffee}(p,s)
\end{equation}
Functional fluents return values instead~\cite{levesque_golog:_1997}. %3
As an example, a fluent $\textit{location}(p,s)$ may return some coordinates $(x,y)$.
This then expresses the agent $p$'s location in situation $s$.

Actions also depend on situations.
The reason for this is that certain actions may only be executed when specific fluents hold.
As fluents are only modified by actions, their result can be determined by the history of action executions contained in the current situation.
Describing when an action is executable is done by \emph{action precondition axioms}~\cite{lin_state_1994}. %655+
This is expressed by the predicate $\textit{Poss}(a,s)$, with $a$ being an action.
As a recurring example, let us think of the ability to pour an agent $p$ coffee.
This must only be possible when $p$ does not already have coffee.
\autoref{a_possPourCoffee} illustrates how this can be formalised.
\begin{equation}\label{a_possPourCoffee}
  \textit{Poss}(\textit{pourCoffee}(p),s) \Leftrightarrow \neg \textit{hasCoffee}(p,s)
\end{equation}

As mentioned before, the execution of any action must alter the situation: $\textit{do}(a,s) \rightarrow s'$.
Its effects on fluents are described by \emph{action effect axioms}.
\autoref{a_effectPourCoffee} shows how pouring a coffee for $p$ will result in $p$ having coffee afterwards.
\begin{equation}\label{a_effectPourCoffee}
  \textit{Poss}(\textit{pourCoffee}(p),s) \rightarrow \textit{hasCoffee}\big(p,\textit{do}(\textit{pourCoffee}(p),s)\big)
\end{equation}
In \autoref{a_effectPourCoffee}, it is unclear whether other fluents are affected by the action execution.
For example, reasoning about $location(p,s')$ would not be possible with $\textit{do}(\textit{pourCoffee}(p,s)) \rightarrow s'$.
This is called the \emph{frame problem} (cf.\ Hayes~\cite{hayes_frame_1971}). %224
Defining for every fluent how every action does or does not affect it is only a theoretical solution.
The reason for that is that the resulting complexity of $\mathcal{O}(A*F)$ would be too high even in most small worlds.
A feasible solution to this problem was proposed by Reiter~\cite{reiter_frame_1991}.
His approach was to define every effect of all actions only once.
Thus, Reiter reduced the complexity to $\mathcal{O}(A*E)$.
This solution is known as the \emph{successor state axiom} and is shown in \autoref{sucStateAxiom}.
\begin{equation}\label{sucStateAxiom}
  \mathit{Poss}(a,s)\rightarrow \big[\mathit{F}(\mathit{do}(a,s)) \Leftrightarrow\gamma_\mathit{F}^+(a,s)\vee\mathit{F}(s)\wedge\neg\gamma_\mathit{F}^-(a,s)\big]
\end{equation}
$\mathit{F}(\mathit{do}(a,s))$ means that the fluent $F$ will be true after executing the action $a$.
The first part of the disjunction is $\gamma_\mathit{F}^+(a,s)$ and expresses that the action made the fluent true.
$\mathit{F}(s)\wedge\neg\gamma_\mathit{F}^-(a,s)$ as the second part expresses that the fluent had been true before and the action had no influence on it.
For a reasonable example, there needs to be a second action which does not influence the fluent given in \autoref{f_hasCoffee}.
Therefore, the $sing(s)$ action will be introduced which has no effect on any fluents and can be executed anytime as shown in \autoref{a_possSing}.
\begin{equation}\label{a_possSing}
  \mathit{Poss}(\mathit{sing}, s) \Leftrightarrow \top
\end{equation}
Given \autoref{f_hasCoffee}, \ref{a_possPourCoffee}, \ref{a_effectPourCoffee} and \ref{a_possSing} an example can be compiled as done in \autoref{a_sucStateAxiom}:
\begin{equation}\label{a_sucStateAxiom}
  \begin{split}
    \mathrm{Poss}(a,s)\rightarrow \big[&\mathrm{hasCoffee}(p,\mathrm{do}(a,s))
\\    &\Leftrightarrow [a=\mathrm{pourCoffee}(p)]
\\    &\vee\ [\mathrm{hasCoffee}(p,s) \wedge a\neq \mathrm{pourCoffee}(p)]\big]
  \end{split}
\end{equation}
\autoref{a_sucStateAxiom} then formalises that an agent $p$ may only have coffee if it was poured coffee or if it already had coffee and the action was not to pour $p$ coffee.

Although the situation calculus contains further concepts, this quick introduction should suffice to get an understanding of it.
\autoref{fun:apl_golog} shows an implementation of these concepts into an agent programming language.


\subsubsection{GOLOG.}\label{fun:apl_golog}
GOLOG is a language for logic programming introduced by Levesque et~al.~\cite{levesque_golog:_1997}. It builds on the situation calculus. To allow high-level programming, GOLOG adds complex actions like loops, conditions, tests and non-deterministic elements. As an example, a GOLOG program should have a robot pouring other agents coffee until everybody does have coffee. After that, the robot should sing and terminate. Such a program would reuse the fluent in \autoref{f_hasCoffee}, the action precondition axioms in \autoref{a_possPourCoffee}, \autoref{a_possSing}, the successor state axiom in \autoref{a_sucStateAxiom} and extend them with the two procedures given in \autoref{p_main} and \autoref{p_pourSOCoffee}:
\begin{equation}\label{p_main}
  \begin{split}
    \textbf{proc}\ \texttt{main}\ [&\textbf{while}\ (\exists p) \neg\textit{hasCoffee}(p) \\
    &\textbf{do}\ \texttt{pourSOCoffee}(p)\ \textbf{endWhile}]; \\
    \textit{sing}&\ \textbf{endProc}.
  \end{split}
\end{equation}
\begin{equation}\label{p_pourSOCoffee}
  \begin{split}
    \textbf{proc}\ \texttt{pourSOCoffee}\ (\boldsymbol{\pi} p)\ [ &\neg\textit{hasCoffee}(p)\textbf{?}; \\
    &\textit{pourCoffee}(p)]\ \textbf{endProc}.
  \end{split}
\end{equation}
\autoref{p_main} shows the procedure which can be seen as the main method. It loops as long as there exist agents without coffee and tells the robot to pour coffee to some agent lacking coffee. In the end, the robot sings. \autoref{p_pourSOCoffee} allows the robot to non-deterministically choose an agent $p$ to pour coffee to by using the $\pi$-operator. The $?$-operator is similar to the \texttt{if}-operator in other programming languages like Java. Due to the non-determinsmic operator, there can be two different resulting situations like shown in \autoref{ex_situations} with the initial configuration given in \autoref{ex_gologConfiguration}:
\begin{equation}\label{ex_gologConfiguration}
  \neg\textit{hasCoffee}(p,s_0) \Leftrightarrow p=\textrm{Jane} \vee p=\textrm{John}.
\end{equation}
\begin{equation}\label{ex_situations}
  \begin{split}
    s=\textit{do}\Big(\textit{sing},\textit{do}\big(&\textit{pourCoffee}(\textrm{Jane}),
      \textit{do}(\textit{pourCoffee}(\textrm{John}),s_0)\big)\Big),
\\  s=\textit{do}\Big(\textit{sing},\textit{do}\big(&\textit{pourCoffee}(\textrm{John}),
      \textit{do}(\textit{pourCoffee}(\textrm{Jane}),s_0)\big)\Big)
  \end{split}
\end{equation}

Levesque et~al.~\cite{levesque_golog:_1997} highlight some problems with GOLOG. These make it unsuitable for a multiple agent-based scenario like the Mars-scenario of the MAPC without considerable modifications and extensions. One problem is that complete knowledge is assumed in the initial situation. This is obviously not the case for scenarios with unknown worlds that get explored by agents. The second problem is that GOLOG does neither offer a solution for internal nor external reactions of agents on sensed actions. A third problem is that exogenous actions say actions out of the agent's control cannot be handled. These could e.g. be actions in control of nature like sudden rain, which are assumed not to be caused by an agent. A fourth problem is highlighted by Thielscher~\cite{thielscher_flux:_2005} and arises from GOLOG being \emph{regression-based}. This means that for deciding whether an action is executable is only possible after looking at all previous actions and how they might have affected the world. As a result, reasoning takes exponentially longer over time and hence GOLOG does not scale.


\subsubsection[FLUX]{FLUX$^\circ$}\label{fun:apl_flux}
This section gives a summary of the logic programming language FLUX which offers solutions to the problems pertaining to GOLOG shown earlier.
Except for the examples and if not specified otherwise, the information of this section is taken from Thielscher~\cite{thielscher_flux:_2005} who first introduced FLUX.
This is done by using the \emph{fluent calculus} instead of the situation calculus.
Both are similar but the fluent calculus adds \emph{states}.
A state $z$ is a set of fluents $f_1,\dotsc,f_n$.
In FLUX, it is denoted as $z = f_1 \circ\dotsc\circ f_n$.
In every situation there exists exactly one state with which the current properties of the world are being described.
Yet, the world can be in the same state in multiple situations.
FLUX uses \emph{knowledge states} for representing agent knowledge.
These are denoted through $\textit{KState}(s,z)$ meaning that an agent knows that $z$ holds in $s$.
Knowledge states can be incomplete as opposed to knowledge in GOLOG.

The frame problem in the fluent calculus is solved through \emph{state update axioms} as described by Thielscher~\cite{thielscher_situation_1999}.
The axioms define the effects of an action as the difference between the state before and after the action.
This is modelled with $\vartheta^-$ for negative and $\vartheta^+$ for positive effects.
Both are simply macros for finite states.
Due to using states, reasoning is linear in the size of the state representation.
That is, after every action execution, the world represented by its fluent is processed.
This is called being \emph{progression-based}.
Therefore, FLUX can outperform GOLOG, as determining whether a property currently holds is only a matter of looking it up in the state.
With GOLOG however, the property must be traced back to the initial situation by looking at all action executions and their effects. %\cite{thielscher_flux:_2005}

Disjunctive and negative state knowledge is modelled through constraints.
FLUX uses a constraint solver to simplify these constraints until they are solvable.
This is done by using \emph{constraint handling rules} introduced by Frühwirth~\cite{fruhwirth_theory_1998}.
Their general form is shown in \autoref{chr}.
It consists of one or multiple heads $H_m$, zero or more guards $G_k$ and one or multiple bodies $B_n$.
The general mechanism is that if the guard can be derived, parts of the constraint matching the head will be replaced by the body and hence get simplified.
\begin{equation}\label{chr}
  H_1,\ldots,H_m\Leftrightarrow G_1,\ldots,G_k \mid B_1,\ldots,B_n
\end{equation}

A FLUX program can be separated into three main parts with the constraint solver building the kernel which is the foundation of a FLUX program.
The domain encodings are built on top of this.
Included are the initial knowledge state(s), domain constraints, as well as the action precondition and state update axioms.
The final part of a FLUX program is the programmer-defined intended agent behaviour, called strategy.
As a trivial example program, the previous example implemented in GOLOG will be transferred to FLUX.
This is done by using the logic programming language Prolog in which FLUX is typically implemented~(cf. \cite{thielscher_reasoning_2006,martin_addressing_2001}). % xi, 1085+, 297
The example features the domain encodings as well as the strategy.
\begin{lstlisting}[caption={Defintion of the \texttt{sing}-action.}, label=lst_sing]
  perform(sing, []).
  poss(sing, Z) :- all_holds(hasCoffee(_), Z).%\label{l_possSing}%
  state_update(Z, sing, Z, []).%\label{l_supSing}%
\end{lstlisting}
\autoref{lst_sing} shows the definition of the \texttt{sing}-action.
Empty arrays denoted by \texttt{[]} could be replaced by sensed information.
They would then effect the outcome of the methods.
As this is a trivial example, no sensed information is assumed.
\autoref{l_possSing} is the precondition that singing is only possible in a state where every agent has coffee.
As singing should not alter any fluents, the state \texttt{Z} in \autoref{l_supSing} is not modified and returned again as \texttt{Z}.
\begin{lstlisting}[firstnumber=4, caption={Definition of the \texttt{pourCoffee}-action}, label=lst_pourCoffee]
  perform(pourCoffee(P), []).
  poss(pourCoffee(P), Z) :-
       member(P,[jane,john]),%\label{l_memberP}%
       not_holds(hasCoffee(P), Z).
  state_update(Z1, pourCoffee(P), Z2, []) :-
       update(Z1, [hasCoffee(P)], [], Z2).%\label{l_updateZ}%
\end{lstlisting}
The \texttt{pourCoffee} action is defined similarly in \autoref{lst_pourCoffee}.
\autoref{l_memberP} ensures that Prolog will only look for agents that actually exist instead of iterating over memory addresses.
The action must modify the state by adding \texttt{hasCoffee(P)} to the state as it is done in \autoref{l_updateZ}.
The array after it corresponds to $\vartheta^-$.
It is empty in this case as no fluents are removed.
\begin{lstlisting}[firstnumber=10, caption={Main method which either tells the robot to sing or to pour coffee.}, label=lst_main]
  main_loop(Z) :-
    poss(sing, Z)
      -> execute(sing, Z, Z);
    poss(pourCoffee(P), Z)
      -> execute(pourCoffee(P), Z, Z1),
         main_loop(Z1);
    false.%\label{l_false}%
\end{lstlisting}
\autoref{lst_main} models the main method and thus is similar to \autoref{p_main}.
When singing is possible, the robot will do so and terminate.
Else, it will pour someone coffee and call the main loop again.
\autoref{l_false} ensures that Prolog will return the false-value \texttt{No} if neither of both actions gets triggered at some point.
\begin{lstlisting}[firstnumber=17, caption={Initial configuration.}, label=lst_init]
  init(Z0) :-
         not_holds(hasCoffee(jane), Z0),
         not_holds(hasCoffee(john), Z0).
\end{lstlisting}
The initial configuration in \autoref{lst_init} is comparable to \autoref{ex_gologConfiguration} but due to Prolog interpreting from top to bottom, the result will be \texttt{Z = [hasCoffee(john), hasCoffee(jane)]}.

Schiffel and Thielscher~\cite{schiffel_multi-agent_2007} successfully applied FLUX to the gold mining domain.
It is a scenario where multiple agents with different roles work together on mining gold in an unknown terrain~\cite{schiffel_multi-agent_2007}.
The requirements for solving the problems arising from this scenario are comparable to those appearing in the \enquote{Agents on Mars} scenario.
Given the former short presentation and this knowledge, it can be said that FLUX could be applied to the \enquote{Agents on Mars} scenario.


\subsubsection[Jadex.]{Jadex.}\label{fun:apl_jadex}


\subsection{AgentSpeak(L).}
This section gives an overview of the general concepts of the logic programming language AgentSpeak(L). The language was developed by Rao~\cite{rao_agentspeak_1996}. Except for the examples, this section takes its information from the given paper. The idea behind AgentSpeak(L) was to make the theoretic concept of BDI-agents usable in practical scenarios. % 44
Therefore, it is applicable to environments where agents affect the world by executing actions, perceive changes in the world and react upon these changes. % TODO: is inter-agent communication a Jason-exclusive feature?

The main language constructs are \emph{beliefs}, \emph{goals} and \emph{plans}. Beliefs represent information that an agent has about its environment. A belief \texttt{hasCoffee(P)} for example denotes that an agent knows that the person \texttt{p} has coffee. In AgentSpeak(L), variables are indicated by using a capital first letter whereas terms with a small first letter are instances. % TODO: can we call these instances? Maybe bound variables? I dunnooo.
\begin{lstlisting}[caption={Initial beliefs.}, label=lst:asl_initBeliefs]
  ~hasCoffee(jane).
  ~hasCoffee(john).
\end{lstlisting}
\autoref{lst:asl_initBeliefs} shows the initial beliefs an agent has for our earlier introduced example. The tilde expresses that the agent knows that neither \texttt{john} nor \texttt{jane} has coffee.

Goals can be dividied into \emph{achievement goals} and \emph{test goals}. The first expresses the wish of an agent to reach a state where a belief holds where the second tests whether a belief holds in the current state. Beliefs hold when the agent knows they are true or when the variables can be bound to at least one known configuration. For example, given an achievement goal !\texttt{hasCoffee}(p)} means that an agent wants to achieve that person \texttt{p} has coffee. Similarly, ?\texttt{hasCoffee}(p)} expresses that an agent tests whether \texttt{p} has a coffee. Hence, this expression will evaluate to true or false depending on the current agent's knowledge.
Achievement goals are comparable to desires. % 45
\autoref{lst:asl_initGoal} shows the initial achievement goal which express that the agent wants to have served everyone coffee.
\begin{lstlisting}[firstnumber=3, caption={Initial goal.}, label=lst:asl_initGoal]
  !servedCoffee.
\end{lstlisting}

\emph{Events} are introduced to allow agents to react on changes in their own knowledge or the world. They can be distinguished into the addition and removal of beliefs or goals. Additions are denoted by a plus- and removals by using a minus-sign in front of the goal or belief: % as shown in \autoref{it:fun:apl_asl}.
\begin{itemize} % \label{it:fun:apl_asl}
  \item $+\textit{hasCoffee}(p)$ an agent is informed that \texttt{p} now has coffee.
  \item $-\textit{hasCoffee}(p)$ an agent is informed that \texttt{p} no longer has coffee.
  \item $+!\textit{hasCoffee}(p)$ an agent is informed that it wants \texttt{p} to have coffee.
  \item $-!\textit{hasCoffee}(p)$ an agent is informed that it no longer wants \texttt{p} to have coffee.
  \item $+?\textit{hasCoffee}(p)$ an agent is informed that it should test for the belief.
  \item $-?\textit{hasCoffee}(p)$ an agent is informed that it no longer needs to test for the belief.
\end{itemize}
In order to handle new events, an agent will look for a matching plan.

Plans can be seen as programmer-defined agent instructions. They lead to the execution of actions or the splitting of goals into additional goals. Plans, which an agent wants to execute, are similar to what are called intentions for BDI-agents. A plan is triggered by events and is context-sensitive. This means that the execution of a plan can be restricted to states in where certain beliefs exist. \autoref{lst:asl_sing} illustrates this by showing when the \texttt{sing}-action is being executed. \autoref{l:asl_trigger} is the triggering event of the plan. In this case, an agent will consider executing this plan, when it notices that someone is poured coffee. Hence, this plan is called a \emph{relevant plan}. The underscore denotes an anonymous variable similar to its use in Prolog. Its meaning is that it will match any term. \autoref{l:asl_context} is the plan's context. The plan is called an \emph{applicable plan} if the context's beliefs are all known to the agent. In this particular case, the agent must know that there is no person without coffee indicated by the use of the tilde. At last, \autoref{l:asl_body} contains the body of the plan. Here, the agent should achieve the goal \texttt{sing}. This will trigger a new event which calls the plan in \autoref{l:asl_sing}. As its context is empty, the plan can be executed immediately and evaluates to true as there is no body. \autoref{l:asl_loop} expresses how the event of someone being poured coffee should be alternatively handled. As AgentSpeak(L) is interpreted from top to bottom, it will only be seen as an applicable plan, if the former relevant plan did not trigger. Therefore, if the agent knew that there was still someone left without coffee, it will want to achieve the \texttt{servedCoffee} goal again.
% TODO: explain that this is a better example and hence we first deal with it instead of servedCoffee
\begin{lstlisting}[firstnumber=4, caption={Events for handling someone being poured a coffee as well as the \texttt{sing} plan.}, label=lst:asl_sing]
  +hasCoffee(_):%\label{l:asl_trigger}%
      ~hasCoffee(_)%\label{l:asl_context}%
      <- !sing.%\label{l:asl_body}%
  +hasCoffee(_)%\label{l:asl_loop}%
      <- !servedCoffee.
  +!sing.%\label{l:asl_sing}%
\end{lstlisting}
% We assume that the implementation of AgentSpeak(L) does not do anything if there is no applicable plan for an event. In \autoref{lst:asl_sing} this would be equal to adding a plan for \texttt{hasCoffee(\_)} without context or body. Given this assumption, the \texttt{pourCoffee}-action is then defined as shown in \autoref{lst:asl_pourCoffee}. \autoref{l:asl_pourCoffee} uses a shortcut operator which extends to \texttt{-hasCoffee(\_); +hasCoffee(X)} \cite{bordini_programming_2007}. % 53
\autoref{lst:asl_serve} contains the plan for serving coffee. It uses an achievement goal to pick someone without a coffee as shown in \autoref{l:asl_thirsty}. The person will be bound to the variable \texttt{X}. After that, an achievement goal is added to the agent's set of intentions to pour \texttt{X} coffee. % TODO introduce set of intentions.
\begin{lstlisting}[firstnumber=10, caption={Definition of the \texttt{servedCoffee} plan.}, label=lst:asl_serve]
  +!servedCoffee:
      <- ?~hasCoffee(X);%\label{l:asl_thirsty}%
         !pourCoffee(X).%\label{l:asl_pour}%
\end{lstlisting}
The plan in \autoref{lst:asl_pour} states that if an agent receives an event to achieve the goal \texttt{!pourCoffee} for some person \texttt{X}, it will pour coffee to \texttt{X}. Additionally, the knowledge about \texttt{X} not having any coffee is removed in \autoref{l:asl_coffeeless}.
\begin{lstlisting}[firstnumber=14, caption={Definition of the \texttt{pourCoffee} plan.}, label=lst:asl_pour]
  +!pourCoffee(X)
      <- +hasCoffee(X);
         -~hasCoffee(X).%\label{l:asl_coffeeless}%
\end{lstlisting}


% NOTE: compared to FLUX, we don't have to manually manipulate the state, which is nice.


\subsubsection[Jason.]{Jason.$^\circ$}\label{fun:apl_jason}
This section gives a quick overview of Jason, which is an interpreter for AgentSpeak(L).
All information if not marked differently is taken from Bordini et al.~\cite{bordini_jason_2005}.
Besides being an interpreter, Jason extends AgentSpeak(L) by several concepts.
The most important ones will be discussed in this section.

With Jason, terms can represent more than a constant or a variable.
They can be strings, integer or floating point numbers or lists of terms.
Therefore, more complex programmatic operations and arithmetic expressions are possible with Jason.
Furthermore, Jason introduces annotations.
With these annotations, metadata can be added to triggering events and beliefs.
This metadata can be accessed programmatically.
\autoref{lst:jason_annotations} shows the earlier used initial beliefs with added annotations.
The \texttt{source} annotation is the only one with its meaning predefined by Jason.
It expresses the source of the information.
If an agent determined something itself, the \texttt{source} is \texttt{self}.
Did the agent receive the information as a perception of the environment, then the \texttt{source} will be \texttt{percept}.
The source can also be a constant identifying a different agent if that agent is the source of this information.
With the example given in \autoref{lst:jason_annotations}, an achievement goal \texttt{?\~{}hasCoffee(X)[reliability(Y)]} will bind \texttt{X} to \texttt{john} and \texttt{Y} to \texttt{0.3}.
The \texttt{reliability} has no further meaning unless the value bound to \texttt{Y} is used later.
\begin{lstlisting}[caption={Annotation of beliefs in Jason.}, label=lst:jason_annotations]
  ~hasCoffee(jane)[source(self)].
  ~hasCoffee(john)[source(percept), reliability(0.3)].
\end{lstlisting}

Another concept added to AgentSpeak(L) by Jason is called \emph{internal actions}.
It was first introduced and implemented by Bordini et al.~\cite{bordini_agentspeak_2002}.
Most characteristic for these actions is that they do not affect the environment in which the agents are located in.
This means they have no effect on the external world but only on the internal states of the agents as the name suggests.
Hence, any effects of internal actions occur immediately after the action execution instead of only after the next environment processing cycle.
As a result, internal actions can not only be used within a plan's body but also in its context. % all this information is from p. 1297
Internal actions start with a dot followed by a library identifier, another dot and finally the action name.
Bordini et al.~\cite{bordini_agentspeak_2002} implemented various internal actions which are not identified by any explicitly named library.
These methods reside in the so called \emph{standard library} and omit the library declaration when being called.
An example for this is \texttt{.gte(X,Y)} which returns the truth value of \texttt{X}$\geq$\texttt{Y}.
A realisation of the same function outside the standard library could e.g. be called \texttt{.math.gte(X,Y)}.
The standard library is included in Jason.
Furthermore, Jason extends this library by various actions including multiple list operations like sorting or retrieving the minimum.
Developers can write additional internal actions in Java or any other programming language which supports the programming framework Java Native Interface. %11

Arguably, the most important  internal action is \texttt{.send}.
This action enables inter-agent communication as initially proposed and implemented by Vierira et al.~\cite{vieira_formal_2007}.
It is structurally based on KQML and FIPA \cite{fernandez_evaluating_2010}.
A short overview of a FIPA message has been given in \autoref{fun:apl_jadex}.
We pass on presenting the structure of a KQML message here as both are similar and KQML is not further developed \cite{obrien_fipatowards_1998}.
\begin{lstlisting}[caption={Parameters of the internal action \texttt{.send} and an example.}, label=lst:jason_send]
  .send(Receiver, Illocutionary_force, Message_content).%\label{l:jason_send}%
  .send([agent1, agent2], tell, ~hasCoffee(john)).%\label{l:jason_sendInstance}%
\end{lstlisting}
In \autoref{l:jason_send} of \autoref{lst:jason_send} the structure of the \texttt{.send} action is shown.
\autoref{l:jason_sendInstance} shows example usage of this action.
The \texttt{Receiver} is the identifying name or a list of identifying names for the agent(s) to which the message should be addressed to.
The \texttt{Illocutionary\_force} is a constant that specifies what all recipients should do with the message.
It can be:
\begin{itemize}
  \item \texttt{tell}: add the \texttt{Message\_content} to the recipient's belief base.
  \item \texttt{untell}: remove the \texttt{Message\_content} from the recipient's belief base.
  \item \texttt{achieve}: add the \texttt{Message\_content} as an achievement goal to the recipient.
  \item \texttt{unachieve}: make the recipient remove the achievement goal \texttt{Message\_content}.
  \item \texttt{tellHow}: \texttt{Message\_content} is added to the recipient's plan library.
  \item \texttt{untellHow}: \texttt{Message\_content} is removed from the recipient's plan library.
  \item \texttt{askIf}: asks if \texttt{Message\_content} is in the recipient's belief base.
  \item \texttt{askOne}: asks for the first belief matching \texttt{Message\_content}.
  \item \texttt{askAll}: asks for all beliefs matching \texttt{Message\_content}.
  \item \texttt{askHow}: demand all plans a recipient has that match the triggering event given in the \texttt{Message\_content}.
\end{itemize}
Jason automatically processes the messages as needed when a message arrives at an agent\todo{should we explicitly talk about an agent's inbox? Compare with @adaudrich's mentioning of the inbox!}.
A developer can override Jason's default behaviour if further or different processing is desired.
Jason also automatically adds \texttt{source} annotations.
This allows agents to determine the sender of any received message.

There is special support for defining environments with Jason.
Instead of having to do this in AgentSpeak(L), it can be done in Java.
For doing so, a developer has to extend the \texttt{Environment} class and specify the \texttt{getPercepts(String agentName)} and \texttt{executeAction(String agentName, Term action)} methods.
The first method must return a list of literals restricted to what the agent identified by \texttt{agentName} can perceive.
When the second method is called, the programmer must specify how the given \texttt{action} affects the environment.
It returns a boolean to indicate whether the execution was successful.
Such an action can fail if for example a repairer agent would try to execute the \texttt{attack} action which it cannot according to the Mars-scenario.
To call the \texttt{executeAction} method from an agent, all it has to do is execute e.g. \texttt{attack}.
Jason will then call \texttt{executeAction(String agentName, Term action)} with the parameters bound to the agent's name and the \texttt{attack} action.
For the MAPC itself, no fully simulated environment is needed.
Instead, it is enough to delegate the actions to the MAPC server and process the server replies by returning the transmitted percepts to the respective agents.
Therefore, percepts do not have to be modelled or modified in the environment developed with Jason itself.

Jason also allows running multi-agent systems over networks in a distributed manner.
Hence, the workload can be distributed over multiple machines.
SACI~\cite{hubner_saci_2000} and JADE are the two fully implemented distributed architectures usable out of the box with Jason \cite{bordini_programming_2007}.
Fernández et al.~\cite{fernandez_evaluating_2010} could not prove the intended performance benefits.
The authors tested both SACI and JADE with Jason where one host would run the environment and the other one the agents.
They increased both the amount of agents as well as the size of the environment.
Fernández et al.~\cite{fernandez_evaluating_2010} saw that with increasing complexity, the system became slower compared to when agents and the environment were run on a single machine.
This was due to the added communication cost between the two hosts although connected by Gigabit Ethernet.
As a result, a distributed infrastructure with Jason is only advisable, if the workload cannot be handled by one host alone.
In our case, replying in time has such an importance that trying to keeping the workload processable by one host alone would be the preferred strategy.


\subsubsection[Choice of a programming language.]{Choice of a programming language.$^{\circ/\odot}$}\label{fun:apl_choice}
Based on the previous sections, this section summarises why we chose Jason for developing our agents.
Generally, we could have started from scratch without using a designated agent programming language.
We decided against this idea because of our inexperience with agent programming and artificial intelligence in general.
The fear was to overlook difficulties in the beginning which would later force us to spend more time on fixing mistakes we made in the beginning than on the actual agent development.
To prevent this, we were interested in using an already developed and approved agent programming language.

Given the Mars scenario, Jason can be used to implement a suitable multi-agent system.
In fact, two teams successfully participated in the 2013 Multi-Agent Programming Contest by using Jason \cite{ahlbrecht_multi_2013}. % p.367
Yet, there was no competing team using Jadex or FLUX.
This is of interest because the scenario of 2013 is comparable to the scenario of 2014 \cite{ahlbrecht_mapc_2014}. % p.1,9
As the whole team was inexperienced with logical programming prior to this research lab, being able to develop the environment and some operations via internal actions in Java was beneficial.
Furthermore, the contest organisers provided a Java library which would simplify the communication with their server.
Instead of having to manually compile XML messages and parse the XML server replies, this library allowed simple method calls for server interaction.
Thus, deciding against FLUX meant not having to implement the communication with the server ourselves.
The library would also have been usable with Jadex.
But just like Flux, Jadex does not assist the developer in modelling the environment like Jason does.
Jason's support for environments allowed us to focus more on agent programming
There, we preferred Jason and Jadex over FLUX, because these two languages are built around BDI, which we found to be a clearer structuring of agents.
FLUX on the other hand serves as a quite generic approach to programming multi-agent systems.
Besides the support for developing environments, Jason and Jadex are also different in the way how the initial beliefs, goals and plans are being programmed.

% TODO integrate the text below:
The difference lies in the storing of beliefs, goals and plans.
In Jadex they are stored in the agent definition file (XML) while in Jason they are stored as facts within the Jason belief base. % TODO @manuelmittler does that mean that the definitions are modified on-the-fly throughout simulation? Because I would have thought, they are initially written in XML and all changes happen in memory. Similarly for Jason, they are written in AS(L)/Java and changes happen in memory as well.
Another slight difference is that in Jadex plans have to be written in Java whereas in Jason the programmer can use a combination of Java and AgentSpeak with internal actions.
We didn't chose Jadex for our research lab because of the overhead that comes with the XML-syntax.

