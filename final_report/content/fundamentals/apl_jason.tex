\subsubsection{Jason.}
This section gives a quick overview of Jason, which is an interpreter for AgentSpeak(L). All information if not marked differently is taken from Bordini et al.~\cite{bordini_jason_2005}. Besides being an interpreter, Jason extends AgentSpeak(L) by a few concepts. One of these concepts is called \emph{internal actions} and was first introduced and implemented by Bordini et al.~\cite{bordini_agentspeak_2002}. Most characteristic for these actions is that they do not affect the environment in which the agents effect. This means they have no effect on the external world but only on the internal states of the agents as the name suggests. Hence, any effects of internal actions occur immediately after the action execution instead of only after the next environment processing cycle. As a result, internal actions can not only be used within a plan's body but also in its context. % all this information is from p. 1297
Internal actions start with a dot follow by a library identifier, another dot and finally the action name. Bordini et al.~\cite{bordini_agentspeak_2002} implemented various internal actions which are not identified by any library explicitly named library. These methods reside in the so called \emph{standard library} and omit the library declaration when being called. An example for this is \texttt{.gte(X,Y)} which returns the truth value of \texttt{X}$\geq$\texttt{Y}. The standard library is included in Jason. Furthermore, Jason extends this library by multiple actions including many list operations like sorting or retrieving the minimum. Developers can write additional internal actions in Java or any other programming language which supports the programming framework Java Native Interface. %11

Arguably, the most important  internal action is \texttt{.send}. This action enables inter-agent communication as initially proposed and implemented by Vierira et al.~\cite{vieira_formal_2007}.
