\subsubsection[Multi-Agent Programming Contest]{Multi-Agent Programming Contest$^{\odot,\circ}$}\label{fun:mapc_general}
The Multi-Agent Programming Contest (short: MAPC) is an international online competition which started in 2005 and took place every year since then.
This tournament is held as an online event in the sense that all participants run their systems locally and communicate with the contest server via Internet.
The winning team is determined by each team playing against all other teams three times.
All participants are free to choose which technologies and implementations they want to use for building their multi-agent system upon.
The scenario of the MAPC has changed several times over the years:
\begin{itemize}
	\item 2005: Food Gatherers
	\item 2006-07: Gold miners
	\item 2008-10: Cows and Cowboys
  \item Since 2011: Agents on Mars
\end{itemize}
The scenarios became more and more complex as the agents' features were extended and heterogeneous agent roles were added~\cite{behrens_multi-agent_2012}. % 113

A detailed description of the \mars{} can be found in Ahlbrecht et~al.~\cite{ahlbrecht_mapc_2014}.
Briefly, mankind finally populated Mars and water wells have been found.
One sub-goal is to occupy the best zones, which means the most valuable water wells, and defend them against the competing team.
The environment is represented as a weighted graph.
Each vertex has a unique identifier and a positive number indicating the value of the well.
Edges have a weight each which represent the costs of traversing this edge.
The environment is unknown in the beginning and thus agents have to explore it.

Zones are determined by using a graph colouring algorithm on the server side which is presented in more detail in Ahlbrecht et~al.~\cite{ahlbrecht_mapc_2014}.
Essentially, a zone is identified as follows:
\begin{enumerate}
	\item A vertex belongs to the team which holds the majority of agents standing on that vertex.
	\item Vertices directly connected to at least two neighbours dominated by the same team also belong to that team.
	\item All vertices that are not reachable by other teams without crossing the already coloured vertices also belong to that team.
    One can see it as a border that	is separating parts of the graph.
\end{enumerate}

The overall goal is to maximise the team's score.
This score is calculated by adding the zones' values and the money for each simulation step as shown in \autoref{eq_score}.
\begin{equation}\label{eq_score}
  \sum_{k=1}^\textit{steps}\left(\textit{zones}_k + \textit{money}_k\right)
\end{equation}
Money is rewarded for achievements like successfully performing an action a certain number of times.
A team has two options for using its money.
It can either be spent to buy upgrades for an agent's attributes or it can be saved in order to increase the score per step.
In the subsequent section, the agents and their attributes are presented in more detail.
