\subsubsection[AgentSpeak(L)]{AgentSpeak(L)$^{\circ,\dagger}$}\label{fun:apl_asl}
This section provides an overview of the general concepts of the logic programming language AgentSpeak(L).
The language was developed by Rao~\cite{rao_agentspeak_1996}.
Except for the examples, this section takes its information from the cited paper.
The idea behind AgentSpeak(L) was to make the theoretic concept of BDI agents usable in practical scenarios. % 44

The main language constructs are \emph{beliefs}, \emph{goals} and \emph{plans}.
Beliefs represent information that an agent has about its environment.
A belief \texttt{hasCoffee(p)} for example denotes that an agent knows that the person \texttt{p} has coffee.
In AgentSpeak(L), variables are indicated by using a capital first letter whereas terms with a lower-case initial letter are constants. % 45
\begin{lstlisting}[caption={Initial beliefs.}, label=lst:asl_initBeliefs]
  ~hasCoffee(jane).
  ~hasCoffee(john).
\end{lstlisting}
\autoref{lst:asl_initBeliefs} shows the initial beliefs of an agent in the example we just introduced.
The tilde ($\sim$) expresses that the agent knows that neither \texttt{john} nor \texttt{jane} has coffee.
At any given time, the sum of all current beliefs of an agent are called its \emph{belief base}~\cite{bordini_jason_2005}. % p.8

Goals can be divided into \emph{achievement goals} and \emph{test goals}.
The first expresses the wish of an agent to reach a state where a belief holds where the second tests whether a belief holds in the current state.
Beliefs hold when the agent knows they are true or when the variables can be bound to at least one known configuration.
For example, a given achievement goal \texttt{!hasCoffee(p)} means that an agent wants to achieve that person \texttt{p} has coffee.
Similarly, \texttt{?hasCoffee(p)} expresses that an agent tests whether \texttt{p} has coffee.
Hence, this expression will evaluate to true or false depending on the agent's knowledge.
Achievement goals are comparable to desires. % 45
\autoref{lst:asl_initGoal} shows the initial achievement goal which express that the agent wants to have served everyone coffee.
\begin{lstlisting}[firstnumber=3, caption={Initial goal.}, label=lst:asl_initGoal]
  !servedCoffee.
\end{lstlisting}

\emph{Events} are introduced to allow agents to react on changes in their own knowledge or the world.
They can be distinguished into the addition and removal of beliefs or goals.
Additions are denoted by a plus (+) and removals by using a minus (-) sign in front of the goal or belief:
\begin{itemize}
  \item \texttt{+hasCoffee(p)}: an agent is informed that \texttt{p} now has coffee.
  \item \texttt{-hasCoffee(p)}: an agent is informed that \texttt{p} no longer has coffee.
  \item \texttt{+!hasCoffee(p)}: an agent is informed that it wants \texttt{p} to have coffee.
  \item \texttt{-!hasCoffee(p)}: an agent is informed that it no longer wants \texttt{p} to have coffee.
  \item \texttt{+?hasCoffee(p)}: an agent is informed that it should test for the belief.
  \item \texttt{-?hasCoffee(p)}: an agent is informed that it no longer needs to test for the belief.
\end{itemize}
In order to handle new events, an agent will look for a matching plan.

Plans can be seen as programmer-defined agent instructions.
They lead to the execution of actions or the splitting of goals into additional goals.
Plans which an agent wants to execute, are similar to intentions for BDI agents.
They are stored as a set of intentions.
The set of plans generally known to an agent is called the \emph{plan library}~\cite{bordini_jason_2005}.
A plan is triggered by events and is context-sensitive.
This means that the execution of a plan can be restricted to states in where certain beliefs exist.
\autoref{lst:asl_sing} illustrates this by showing when the \texttt{sing} action is being executed.
\autoref{l:asl_trigger} is the triggering event of the plan.
In this case, an agent will consider executing this plan, when it notices that someone is poured coffee.
Hence, this plan is called a \emph{relevant plan}.
The underscore (\texttt{\_}) denotes an anonymous variable similar to its use in Prolog.
Its meaning is that it will match any term.
\autoref{l:asl_context} is the plan's context.
The plan is called an \emph{applicable plan} if the context's beliefs are all known to the agent.
In this particular case, the agent must know that there is no person without coffee indicated by the use of the tilde.
Lastly, \autoref{l:asl_body} contains the body of the plan.
Here, the agent should achieve the goal \texttt{sing}.
This will trigger a new event which calls the plan in \autoref{l:asl_sing}.
As its context is empty, the plan can be executed immediately and evaluates to true as there is no body.
\autoref{l:asl_loop} expresses how the event of someone being poured coffee should be alternatively handled.
As AgentSpeak(L) is interpreted from top to bottom, it will only be seen as an applicable plan, if the former relevant plan did not trigger.
Therefore, if the agent knew that there was still someone left without coffee, it will want to achieve the \texttt{servedCoffee} goal again.
% TODO?: explain that this is a better example and hence we first deal with it instead of servedCoffee
\begin{lstlisting}[firstnumber=4, caption={Events for handling someone being poured a coffee as well as the \texttt{sing} plan.}, label=lst:asl_sing]
  +hasCoffee(_):%\label{l:asl_trigger}%
      ~hasCoffee(_)%\label{l:asl_context}%
      <- !sing.%\label{l:asl_body}%
  +hasCoffee(_)%\label{l:asl_loop}%
      <- !servedCoffee.
  +!sing.%\label{l:asl_sing}%
\end{lstlisting}
% We assume that the implementation of AgentSpeak(L) does not do anything if there is no applicable plan for an event. In \autoref{lst:asl_sing} this would be equal to adding a plan for \texttt{hasCoffee(\_)} without context or body. Given this assumption, the \texttt{pourCoffee}-action is then defined as shown in \autoref{lst:asl_pourCoffee}. \autoref{l:asl_pourCoffee} uses a shortcut operator which extends to \texttt{-hasCoffee(\_); +hasCoffee(X)} \cite{bordini_programming_2007}. % 53
\autoref{lst:asl_serve} contains the plan for serving coffee.
It uses a test goal to pick someone without a coffee as shown in \autoref{l:asl_thirsty}.
The person will be bound to the variable \texttt{X}.
After that, an achievement goal is added to the agent's set of intentions to pour \texttt{X} coffee.
The plan does not feature any context as this minimal example ensures that the goal \texttt{!servedCoffee} will only exist when there actually is a person without coffee.
\begin{lstlisting}[firstnumber=10, caption={Definition of the \texttt{servedCoffee} plan.}, label=lst:asl_serve]
  +!servedCoffee:
      <- ?~hasCoffee(X);%\label{l:asl_thirsty}%
         !pourCoffee(X).%\label{l:asl_pour}%
\end{lstlisting}
\autoref{lst:asl_pour} shows a plan which states that if an agent receives an event to achieve the goal \texttt{!pourCoffee} for some person \texttt{X}, it will pour coffee for \texttt{X}.
Additionally, the knowledge about \texttt{X} not having any coffee is removed in \autoref{l:asl_coffeeless}.
\begin{lstlisting}[firstnumber=14, caption={Definition of the \texttt{pourCoffee} plan.}, label=lst:asl_pour]
  +!pourCoffee(X)
      <- +hasCoffee(X);
         -~hasCoffee(X).%\label{l:asl_coffeeless}%
\end{lstlisting}

As shown, AgentSpeak(L) is another logic programming language for agent programming.
It was specifically designed for developing BDI-agents.
In the next section, an interpreter for this language will be given, which further extends it.
