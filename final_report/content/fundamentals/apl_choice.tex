\subsubsection[Choice of a programming language.]{Choice of a programming language.$^{\circ/\odot}$}\label{fun:apl_choice}
Based on the previous sections, this section summarises why we chose Jason for developing our agents.
Generally, we could have started from scratch without using a designated agent programming language.
We decided against this idea because of our inexperience with agent programming and artificial intelligence in general.
The fear was to overlook difficulties in the beginning which would later force us to spend more time on fixing mistakes we made in the beginning than on the actual agent development.
To prevent this, we were interested in using an already developed and approved agent programming language.

Given the Mars scenario, Jason can be used to implement a suitable multi-agent system.
In fact, two teams successfully participated in the 2013 Multi-Agent Programming Contest by using Jason \cite{ahlbrecht_multi_2013}. % p.367
Yet, there was no competing team using Jadex or FLUX.
This is of interest because the scenario of 2013 is comparable to the scenario of 2014 \cite{ahlbrecht_mapc_2014}. % p.1,9
As the whole team was inexperienced with logical programming prior to this research lab, being able to develop the environment and some operations via internal actions in Java was beneficial.
Furthermore, the contest organisers provided a Java library which would simplify the communication with their server.
Instead of having to manually compile XML messages and parse the XML server replies, this library allowed simple method calls for server interaction.
Thus, deciding against FLUX meant not having to implement the communication with the server ourselves.
The library would also have been usable with Jadex.
But just like Flux, Jadex does not assist the developer in modelling the environment like Jason does.
Jason's support for environments allowed us to focus more on agent programming
There, we preferred Jason and Jadex over FLUX, because these two languages are built around BDI, which we found to be a clearer structuring of agents.
FLUX on the other hand serves as a quite generic approach to programming multi-agent systems.
Besides the support for developing environments, Jason and Jadex are also different in the way how the initial beliefs, goals and plans are being programmed.

% TODO integrate the text below:
The difference lies in the storing of beliefs, goals and plans.
In Jadex they are stored in the agent definition file (XML) while in Jason they are stored as facts within the Jason belief base. % TODO @manuelmittler does that mean that the definitions are modified on-the-fly throughout simulation? Because I would have thought, they are initially written in XML and all changes happen in memory. Similarly for Jason, they are written in AS(L)/Java and changes happen in memory as well.
Another slight difference is that in Jadex plans have to be written in Java whereas in Jason the programmer can use a combination of Java and AgentSpeak with internal actions.
We didn't chose Jadex for our research lab because of the overhead that comes with the XML-syntax.
