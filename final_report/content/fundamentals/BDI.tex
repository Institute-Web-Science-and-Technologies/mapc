
\section{Scientific Background and Fundamentals}
\subsection{BDI Model}
Using cognitive modelling techniques for simulating human behaviour, without requiring people interactions can save a lot of people forces, resources, time and money. So that a variety of researchers are contributing to intelligent agents field. Beliefs, desires and intentions (BDI) agent is a kind of intelligent agent. BDI model describes the basic characteristics of agents' mental state since the BDI logic system is easy to be realized in the computer, and has been wildly applied in these fields. In recent years, many scholars have used Java, Jason or some other languages to implement BDI agent model in computer. 

In 1987,Bratman\cite{MICHAEL_PlansResource_1988} discussed the relationship among beliefs, desires, intentions and actions as well as considering that they play important roles in option behaviours. This is the foundation of BDI model and BDI logic. In 1991, Rao and Georgeff\cite{Michael_BDIAgency_1999} modelled the BDI agent behaviour and treated beliefs, desires and intentions as three modal operators and applied BDI agent to airline traffic management. Nowadays,the research on BDI agents are not only used in high value domains but also in daily lives. We can see the applications are not only in high technology industrial aspect as air-plane or space shuttle but also in commercial field or entertainment such as robot soccer games.

The BDI model is a popular and well-studied architecture of agent for intelligent agents situated in complex and dynamic environments. The model has its roots in philosophy with Bratman’s theory of practical reasoning\cite{Sebastian_Hierarchical_2006}. Practical reasoning involves two important processes: deciding what goals we want to achieve, and how we are going to achieve these goals. The former process is known as deliberation, the latter as means-ends reasoning\cite{Gerhard_MultiSystem_1999}. When an agent is placed in an environment, it should decide what to do and how to do. There are a lot of options of affairs states, but not all of them are good choices. Some other affairs more or less have influences on the feasibility of achieving these goals. The deliberation process is to understand and filter what options are available, in addition, generate the set of alternatives which will be chosen as following. These chosen options become intentions which can be treated as the outputs of deliberation. For example, if you are standing in a supermarket and very thirsty, then you are faced with a decision to choose a drink. There are a lot of options like wine, beers, milk, water and juice, however, picking up a bottle of wine is not available to you if you are younger than 18 years old. After collecting all the available options, you must choose and commit to some of them which become intentions next. Subsequently,we need the mean-ends reasoning process to plan how to achieve these intentions. Furthermore, your intention is to buy a bottle of water, then you plan to go to the shelf with water on it, and stretch your arm to get a bottle of water on the top. Finally, you execute this plan to get a water.

As a theory of practical reasoning, BDI model has three attributes that are belief, desire and intention.

%TODo BDI three attributes
%ToDo BDI basic architecture//pic
%ToDo BDI applications (short)


