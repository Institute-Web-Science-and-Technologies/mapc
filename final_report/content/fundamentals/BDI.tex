
\section{Scientific Background and Fundamentals}
\subsection{BDI Model}
Using cognitive modelling techniques for simulating human behaviour, without requiring people interactions can save a lot of people forces, resources, time and money. So that a variety of researchers are contributing to intelligent agents field. Beliefs, desires and intentions (BDI) agent is a kind of intelligent agent. BDI model describes the basic characteristics of agents' mental state since the BDI logic system is easy to be realized in the computer, and has been wildly applied in these fields. In recent years, many scholars have used Java, Jason or some other languages to implement BDI agent model in computer. 

In 1987,Bratman\cite{MICHAEL_PlansResource_1988} discussed the relationship among beliefs, desires, intentions and actions as well as considering that they play important roles in option behaviours. This is the foundation of BDI model and BDI logic. In 1991, Rao and Georgeff\cite{Michael_BDIAgency_1999} modelled the BDI agent behaviour and treated beliefs, desires and intentions as three modal operators and applied BDI agent to airline traffic management. Nowadays,the research on BDI agents are not only used in high value domains but also in daily lives. We can see the applications are not only in high technology industrial aspect as air-plane or space shuttle but also in commercial field or entertainment such as robot soccer games.

The BDI model is a popular and well-studied architecture of agent for intelligent agents situated in complex and dynamic environments. The model has its roots in philosophy with Bratman’s theory of practical reasoning\cite{Sebastian_Hierarchical_2006}. Practical reasoning involves two important processes: deciding what goals we want to achieve, and how we are going to achieve these goals. The former process is known as deliberation, the latter as means-ends reasoning\cite{Gerhard_MultiSystem_1999}. When an agent is placed in an environment, it should decide what to do and how to do. There are a lot of options of affairs states, but not all of them are good choices. Some other affairs more or less have influences on the feasibility of achieving these goals. The deliberation process is to understand and filter what options are available, in addition, generate the set of alternatives which will be chosen as following. These chosen options become intentions which can be treated as the outputs of deliberation. For example, if you are standing in a supermarket and very thirsty, then you are faced with a decision to choose a drink. There are a lot of options like wine, beers, milk, water and juice, however, picking up a bottle of wine is not available to you if you are younger than 18 years old. After collecting all the available options, you must choose and commit to some of them which become intentions next. Subsequently,we need the mean-ends reasoning process to plan how to achieve these intentions. Furthermore, your intention is to buy a bottle of water, then you plan to go to the shelf with water on it, and stretch your arm to get a bottle of water on the top. Finally, you execute this plan to get a water.

As a theory of practical reasoning, BDI model has three attributes that are belief, desire and intention.

%TODo BDI three attributes
Beliefs represent the informational state of the agent and be updated appropriately after each sensing action.They may be implemented as a variable, a database, a set of logical expressions, or some other data structure\cite{Rao_BDITheory_1995}. Belief means how the agent look at the world and it is the basis of BDI model. Belief includes the information about environment, other agents and itself. An agent needs to be allowed to update its beliefs at any time. Updating information comes from the perception of the environment, and the execution of intentions. An agent can use sensors to perceive the environment to get signals to believe. In addition, after executing some intentions, these become the information believed by the agent. Belief is not the same concept as knowledge. Beliefs are only required to provide information on the likely state of the environment, but knowledge is the realization of a fact. Beliefs are just the state believed by agents but no one can ensure what they believe are true. Simple to say, knowledge is true belief. 

Desires represent the motivational state of the agent\cite{Rao_BDITheory_1995}. They represent objectives or situations that the agent would like to accomplish or bring about. They are state of affairs that the agent would wish to bring about or to keep. Desires may be achieved or never achieved, and it doesn't need to believe that desires must be achieved. Desires are different from goals although they look pretty similar. Desires can be inconsistent and the agent need not know the means of achieving these desires. Desires have the tendency to 'tug' the agent in different directions. They are inputs to the agent's deliberation process, which results in the agent choosing a subset of desires that are both consistent and achievable. Such consistent achievable desires are usually called goals\cite{Gerhard_MultiSystem_1999}. For example, sleeping and working may be both my desires, but they can not be my goals at the same time because they have conflicts.

Intentions are desires or actions that the agent has committed to achieve\cite{Alejandro_LearnBDI_2004}. Intentions are stronger than desires. Desires are just wishes that may be achieved or may be not, but intentions to an extent are decided to be achieved. Michael Wooldridge concluded four roles of intentions playing in practical reasoning. The roles are intentions drive means-ends reasoning; intentions constrain future deliberation; intentions persist and intentions influence beliefs upon which future practical reasoning is based\cite{Gerhard_MultiSystem_1999}. Intentions driving means-ends reasoning means that intentions have decisive influences on actions the agent will execute. Agents are expected to determine ways of achieving intentions. Intentions constraining future deliberation means that options that are inconsistent with this intention will not be entertained. intentions persisting means that intentions will be never given up unless the reason is rational.As we know,intentions are committed desires which can not be easily abandoned. For if I immediately drop my intentions without devoting resources to achieving them, then I will never achieve anything\cite{Gerhard_MultiSystem_1999}. But when a good reason exists, I still can drop intentions instead of persisting them for a long time without achievement. If it is very clear that the intentions will never been achieved,then there is no need to keep them. Similarly, if the reason for intention is no longer true,then intentions should be given up. Another reason of dropping intentions is the intentions have been achieved already. Intentions influencing beliefs upon which future practical reasoning is based means that believing intentions will be achieved. If I adopt an intention, then I can plan for the future on the assumption that I will achieve the intention. For if I intend to achieve some state of affairs while simultaneously believing that I will not achieve it, then I am being irrational\cite{Gerhard_MultiSystem_1999}. Agents should believe that they believe there is at least some way that the intentions could be brought about and believe that under "normal circumstances" agents will succeed with the intentions, or say it in another way, agents do not believe they will not bring about their intentions. Generally speaking, intentions are not random ideas, but the wants to a reasonable extent. It plays an important role in BDI model that leading to actions, constraining future deliberation and influencing future beliefs. Specifically,the agents should drop off some intentions at times to avoid resource wasting. It is necessary to keep a good balance between these different concerns.

Beliefs, desires and intentions are three attributes of BDI model and constitute the foundation of BDI agents. While, some other components building connection between beliefs, desires and intentions are also indispensable to implement BDI agents and make the BDI architecture completed.
%ToDo BDI basic architecture//pic
%ToDo BDI applications (short)


