\subsubsection{Situation Calculus.}\label{fun:apl_sitCalc}
The situation calculus was introduced by McCarthy and Hayes~\cite{mccarthy_philosophical_1969}. It is mainly a fist-order logic but also uses second order logic to encode a dynamic world \cite{levesque_golog:_1997}. The situation calculus consists of the three first-order terms: \emph{fluents}, \emph{actions} and \emph{situations}. Fluents model properties of the world. Actions may change fluents and hence may modify the world. Every action execution creates a new situation. A situation is a history of actions up to a certain point in time starting from the initial situation $s_0$. There can only be one initial situation as it models the situation before any action has been executed.

Fluents can be evaluated to return a result. As they are situation dependent, the evaluation result may change over time. Fluents are distinguished in \emph{relational fluents} and \emph{functional fluents}. Relational fluents can hold in situations. Their evaluation hence may return either true or false. An example is given in \autoref{f_hasCoffee}. It expresses whether or not the agent $p$ has a coffee in situation $s$.
\begin{equation}\label{f_hasCoffee}
  \textit{hasCoffee}(p,s)
\end{equation}
Functional fluents return values instead. As an example, a fluent $\textit{location}(p,s)$ may return some coordinates $(x,y)$ This then expresses the agent $p$'s location in situation $s$.

Actions also depend on situations. The reason for this is that certain actions may only be executed when specific fluents hold. As fluents are only modified by actions, their result can be determined by the history of action executions contained in the current situation. Describing when an action is executable is done by \emph{action precondition axioms}. This is expressed by the predicate $\textit{Poss}(a,s)$ with $a$ being an action. As a recurring example, let us think of the ability to pour an agent $p$ coffee. This must only be possible when $p$ does not already have coffee. \autoref{a_possPourCoffee} illustrates how this can be formalised.
\begin{equation}\label{a_possPourCoffee}
  \textit{Poss}(\textit{pourCoffee}(p),s) \Leftrightarrow \neg \textit{hasCoffee}(p,s)
\end{equation}

As mentioned before, the execution of any action must alter the situation: $\textit{do}(a,s) \rightarrow s'$. Its effects on fluents are described by \emph{action effect axioms}. \autoref{a_effectPourCoffee} shows how pouring a coffee to $p$ will result in $p$ having coffee afterwards.
\begin{equation}\label{a_effectPourCoffee}
  \textit{Poss}(\textit{pourCoffee}(p),s) \rightarrow \textit{hasCoffee}\big(p,\textit{do}(\textit{pourCoffee}(p),s)\big)
\end{equation}
In \autoref{a_effectPourCoffee}, it is unclear whether other fluents are affected by the action execution. For example, reasoning about $location(p,s')$ would not be possible with $\textit{do}(\textit{pourCoffee}(p,s)) \rightarrow s'$. This is called the \emph{frame problem}. Defining for every fluent how every action does or does not affect it is only a theoretical solution. The reason for that is that the resulting complexity of $\mathcal{O}(A*F)$ would be too high even in most small worlds. A feasible solution to this problem was proposed by Reiter~\cite{reiter_frame_1991}. His approach was to define every effect of all actions only once. Thus, Reiter reduced the complexity to $\mathcal{O}(A*E)$. This solution is known as the \emph{successor state axiom} and is shown in \autoref{sucStateAxiom}.
\begin{equation}\label{sucStateAxiom}
  \mathit{Poss}(a,s)\rightarrow \big[\mathit{F}(\mathit{do}(a,s)) \Leftrightarrow\gamma_\mathit{F}^+(a,s)\vee\mathit{F}(s)\wedge\neg\gamma_\mathit{F}^-(a,s)\big]
\end{equation}
$\mathit{F}(\mathit{do}(a,s))$ means that the fluent $F$ will be true after executing the action $a$. The first part of the disjunction is $\gamma_\mathit{F}^+(a,s)$ and expresses that the action made the fluent true. $\mathit{F}(s)\wedge\neg\gamma_\mathit{F}^-(a,s)$ as the second part expresses that the fluent had been true before and the action had no influence on it. For a reasonable example, there needs to be a second action which does not influence the fluent given in \autoref{f_hasCoffee}. Therefore, the $sing(s)$ action will be introduced which has no effect on any fluents and can be executed anytime as shown in \autoref{a_possSing}.
\begin{equation}\label{a_possSing}
  \mathit{Poss}(\mathit{sing}, s) \Leftrightarrow \top
\end{equation}
Given \autoref{f_hasCoffee}, \ref{a_possPourCoffee}, \ref{a_effectPourCoffee} and \ref{a_possSing} an example can be compiled as done in \autoref{a_sucStateAxiom}:
\begin{equation}\label{a_sucStateAxiom}
  \begin{split}
    \mathrm{Poss}(a,s)\rightarrow \big[&\mathrm{hasCoffee}(p,\mathrm{do}(a,s))
\\    &\Leftrightarrow [a=\mathrm{pourCoffee}(p)]
\\    &\vee\ [\mathrm{hasCoffee}(p,s) \wedge a\neq \mathrm{pourCoffee}(p)]\big]
  \end{split}
\end{equation}
\autoref{a_sucStateAxiom} then formalises that an agent $p$ may only have coffee if it was poured coffee or if it already had coffee and the action was not to pour $p$ a coffee.
