\subsubsection[GOLOG]{GOLOG$^{\circ,\dagger}$}\label{fun:apl_golog}
This section gives a summary of the logic programming language GOLOG.
If not further specified, all information except for the examples is taken from Levesque et~al.~\cite{levesque_golog:_1997} who introduced the language.
GOLOG builds on the situation calculus.
To allow high-level programming, the language adds complex actions like loops, conditions, tests and non-deterministic elements.
As an example, a GOLOG program should have a robot pouring other agents coffee until everybody has coffee.
After that, the robot should sing and terminate.
Such a program would reuse the fluent of \autoref{f_hasCoffee}, the action precondition axioms of \autoref{a_possPourCoffee} and \ref{a_possSing}, the successor state axiom of \autoref{a_sucStateAxiom} and extend them with the two procedures given in \autoref{p_main} and \ref{p_pourSOCoffee}:
\begin{equation}\label{p_main}
  \begin{split}
    \textbf{proc}\ \texttt{main}\ [&\textbf{while}\ (\exists p) \neg\textit{hasCoffee}(p) \\
    &\textbf{do}\ \texttt{pourSOCoffee}(p)\ \textbf{endWhile}]; \\
    \textit{sing}&\ \textbf{endProc}.
  \end{split}
\end{equation}
\begin{equation}\label{p_pourSOCoffee}
  \begin{split}
    \textbf{proc}\ \texttt{pourSOCoffee}\ (\boldsymbol{\pi} p)\ [ &\neg\textit{hasCoffee}(p)\textbf{?}; \\
    &\textit{pourCoffee}(p)]\ \textbf{endProc}.
  \end{split}
\end{equation}
\autoref{p_main} shows the procedure which can be seen as the main method.
It loops as long as there exist agents without coffee and tells the robot to pour coffee for some agent which is lacking coffee.
After completion of its coffee-pouring task, the robot sings.
\autoref{p_pourSOCoffee} allows the robot to non-deterministically choose an agent $p$ to pour coffee for by using the $\pi$-operator.
The $?$-operator is similar to the \texttt{if}-operator in other programming languages like Java.
Due to the non-determinsmic operator, there can be two different resulting situations as shown in \autoref{ex_situations} with the initial configuration given in \autoref{ex_gologConfiguration}:
\begin{equation}\label{ex_gologConfiguration}
  \neg\textit{hasCoffee}(p,s_0) \Leftrightarrow p=\textrm{Jane} \vee p=\textrm{John}.
\end{equation}
\begin{equation}\label{ex_situations}
  \begin{split}
    s=\textit{do}\Big(\textit{sing},\textit{do}\big(&\textit{pourCoffee}(\textrm{Jane}),
      \textit{do}(\textit{pourCoffee}(\textrm{John}),s_0)\big)\Big),
\\  s=\textit{do}\Big(\textit{sing},\textit{do}\big(&\textit{pourCoffee}(\textrm{John}),
      \textit{do}(\textit{pourCoffee}(\textrm{Jane}),s_0)\big)\Big)
  \end{split}
\end{equation}

GOLOG is \emph{regression-based}.
This means that deciding whether a property holds in a situation requires looking at all prior changes to it.
For this, the property must be traced back to the initial situation by looking at all action executions and their effects.
Depending on how often this needs to be done and how many actions have already been executed, reasoning can take a long time.

As shown, GOLOG transfers the earlier presented concepts of the situation calculus into a logic programming language.
It enables the comfortable use of control flow statements like \texttt{while}-loops.
These statements are macros which a GOLOG interpreter expands into solvable formulas.
Levesque~et~al.~\cite{levesque_golog:_1997} provide such an interpreter written in Prolog in their paper.
The next section introduces another logic programming language FLUX which is based on a different calculus.
In the later \autoref{fun:apl_choice}, critical differences between FLUX and GOLOG for applying them to a multi-agent scenario are discussed.
