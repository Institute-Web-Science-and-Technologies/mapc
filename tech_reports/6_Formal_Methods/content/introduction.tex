Deploying of autonomous agents becoming more and more important nowadays. Agents act in complex production environments, where failure of a single agent may cause serious losses. The challenge for multi-agent systems now is how to make sure that the agent will not behave unacceptable or undesirable? Formal methods had been used in computer science as a basis to solve correctness challenges. They represent agents as a high level abstractions in complex systems. Such representation can lead to simpler techniques for design and development.

There are two roles of formal methods in distributed artificial intelligence that are often referred to. Firstly, with respect to precise specifications they help in debugging specifications and in validation of system implementations. Abstracting from specific implementation leads to better understanding of the design of the system being developed. Secondly, in the long run formal methods help in developing a clearer understanding of problems and their solutions. \cite{Singh_99}

This report in the first section will cover very briefly the theoretical background consisting of different types of logics and introduced operators. Later we will discuss formal methods for a single autonomous agent and the basic implementation of the interpreter. Next some concepts for multiple communicating agents will be introduced. And, finally, the report will conclude with a brief summary of the discussed topic. 